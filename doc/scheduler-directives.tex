% ------------------------------------------------------------------------------
\subsubsection{Directives}
\label{sect:directives}

Directives are comments included at the beginning of a job script that set the shell 
and the options for the job scheduler. 
%
The shebang directive is always the first line of a script. In your job script, 
this directive sets which shell your script's commands will run in. On ``Speed'', 
we recommend that your script use a shell from the \texttt{/encs/bin} directory. 

To use the \texttt{tcsh} shell, start your script with \verb|#!/encs/bin/tcsh|.
%
For \texttt{bash}, start with \verb|#!/encs/bin/bash|.
%
Directives that start with \verb|#SBATCH|, set the options for the cluster's 
SLURM scheduler. The script template, \texttt{template.sh}, 
provides the essentials:

%\begin{verbatim}
%#$ -N <jobname>
%#$ -cwd
%#$ -m bea
%#$ -pe smp <corecount>
%#$ -l h_vmem=<memory>G
%\end{verbatim}
\begin{verbatim}
#SBATCH --job-name=tmpdir           ## Give the job a name
#SBATCH --mail-type=ALL             ## Receive all email type notifications
#SBATCH --mail-user=$USER@encs.concordia.ca
#SBATCH --chdir=./                  ## Use current directory as working directory
#SBATCH --nodes=1
#SBATCH --ntasks=1
#SBATCH --cpus-per-task=<corecount> ## Request, e.g. 8 cores
#SBATCH --mem=<memory>              ## Assign, e.g., 32G memory per node 
\end{verbatim}

and its short option equivalents:

\begin{verbatim}
#SBATCH -J tmpdir                   ## Give the job a name
#SBATCH --mail-type=ALL             ## Receive all email type notifications
#SBATCH --mail-user=$USER@encs.concordia.ca
#SBATCH --chdir=./                  ## Use current directory as working directory
#SBATCH -N 1
#SBATCH -n 8                        ## Request 8 cores
#SBATCH --mem=32G                   ## Assign 32G memory per node 
\end{verbatim}

Replace, \verb+<jobname>+, with the name that you want your cluster job to have;
\option{--chdir}, makes the current working directory the ``job working directory'',
and your standard output file will appear here; \option{--mail-type}, provides e-mail
notifications (success, error, etc. or all); replace, \verb+<corecount>+, with the degree of
(multithreaded) parallelism (i.e., cores) you attach to your job (up to 32 by default).
%be sure to delete or comment out the \verb| #$ -pe smp | parameter if it 
%is not relevant;

Replace, \verb+<memory>+, with the value (in GB), that you want 
your job's memory space to be (up to 500 depending on the node), and all jobs MUST have a memory-space 
assignment.
%
If you are unsure about memory footprints, err on assigning a generous
memory space to your job, so that it does not get prematurely terminated.
%(the value given to \api{h\_vmem} is a hard memory ceiling).
You can refine
%\api{h\_vmem}
\option{--mem}
values for future jobs by monitoring the size of a job's active
memory space on \texttt{speed-submit} with:

%\begin{verbatim}
%qstat -j <jobID> | grep maxvmem
%\end{verbatim}

\begin{verbatim}
sacct -j <jobID>
sstat -j <jobID>
\end{verbatim}

\noindent
This can be customized to show specific columns:

\begin{verbatim}
sacct -o jobid,maxvmsize,ntasks%7,tresusageouttot%25 -j <jobID>
sstat -o jobid,maxvmsize,ntasks%7,tresusageouttot%25 -j <jobID>
\end{verbatim}

Memory-footprint values are also provided for completed jobs in the final
e-mail notification (as, ``maxvmsize'').
%
\emph{Jobs that request a low-memory footprint are more likely to load on a busy
cluster.}

Other essential options are \option{-t} and \option{-A}.
%
\begin{itemize}
\item
\option{-t} -- is the time estimate how long your job may run. This is
used in scheduling priority of your job. The maximum already mentioned
is 7 days for batch and 24 hours for interactive. Specifying lesser
time may have your job scheduled sooner. The ``best'' value for this
does not exist and is often determined empirically from the past runs.

\item
\option{-A} -- to what projects/associations attribute the accounting to. This is usually
your research or supervisor group or a project or some kind of
association. When moving from GE to SLURM we ported most users to
two default accounts \texttt{speed1} and \texttt{speed2}. These
are generic catch-all accounts if you are unsure what to use.
Normally we tell in our intro email which one to use, which may
be your default account. For example,
\texttt{aits},
\texttt{vidpro},
\texttt{gipsy},
\texttt{ai2},
\texttt{mpackir},
\texttt{cmos}, among others.

\end{itemize}
