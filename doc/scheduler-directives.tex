%  2.2.1 Directives
% -------------------
% TMP scheduler-specific section

Directives are comments included at the beginning of a job script that set the shell and the options for the job scheduler. 
The shebang directive is always the first line of a script. 
In your job script, this directive sets which shell your script's commands will run in. On ``Speed'', 
we recommend that your script use a shell from the \texttt{/encs/bin} directory.

To specify the shell for your script, begin with the appropriate shebang line:\\
use \verb|#!/encs/bin/tcsh| for the tcsh shell, or \verb|#!/encs/bin/bash| for bash.

Directives that start with \verb|#SBATCH| set the options for the cluster's 
SLURM job scheduler. The following provides an example of some essential directives:

\small
\begin{verbatim}
    #SBATCH --job-name=<jobname>        ## or -J. Give the job a name 
    #SBATCH --mail-type=<type>          ## set type of email notifications
    #SBATCH --chdir=<directory>         ## or -D, set working directory for the job
    #SBATCH --nodes=1                   ## or -N, node count required for the job
    #SBATCH --ntasks=1                  ## or -n, number of tasks to be launched
    #SBATCH --cpus-per-task=<corecount> ## or -c, core count requested, e.g. 8 cores
    #SBATCH --mem=<memory>              ## assign memory for this job,
                                        ## e.g., 32G memory per node 
\end{verbatim}
\normalsize

\noindent Replace the following to adjust the job script for your project(s)
\begin{itemize}
    \item \verb+<jobname>+ with a job name for the job. This name will be displayed in the job queue.
    \item \verb+<directory>+ with the fullpath to your job's working directory, e.g., where your code, 
    source files and where the standard output files will be written to. 
    By default, \verb+--chdir+ sets the current directory as the job's working directory.
    \item \verb+<type>+ with the type of e-mail notifications you wish to receive. 
    Valid options are: NONE, BEGIN, END, FAIL, REQUEUE, ALL.
    \item \verb+<corecount>+ with the degree of multithreaded parallelism (i.e., cores) allocated to your job. Up to 32 by default.
    \item \verb+<memory>+ with the amount of memory, in GB, that you want to be allocated per node. Up to 500 depending on the node.\\
    \textbf{Note}: All jobs MUST set a value for the \option{--mem} option.
\end{itemize}

\noindent Example with short option equivalents:
\small
\begin{verbatim}
    #SBATCH -J myjob              ## Job's name set to 'myjob'
    #SBATCH --mail-type=ALL       ## Receive all email type notifications
    #SBATCH -D ./                 ## Use current directory as working directory
    #SBATCH -N 1                  ## Node count required for the job
    #SBATCH -n 1                  ## Number of tasks to be launched
    #SBATCH -c 8                  ## Request 8 cores
    #SBATCH --mem=32G             ## Allocate 32G memory per node 
\end{verbatim}
\normalsize

\noindent \textbf{Tip:} If you are unsure about memory footprints, err on assigning a generous
memory space to your job, so that it does not get prematurely terminated.
You can refine \option{--mem} values for future jobs by monitoring the size of a job's active
memory space on \texttt{speed-submit} with:

\begin{verbatim}
    sacct -j <jobID>
    sstat -j <jobID>
\end{verbatim}

\noindent This can be customized to show specific columns:

\begin{verbatim}
    sacct -o jobid,maxvmsize,ntasks%7,tresusageouttot%25 -j <jobID>
    sstat -o jobid,maxvmsize,ntasks%7,tresusageouttot%25 -j <jobID>
\end{verbatim}

Memory-footprint efficiency values (\tool{seff}) are also provided for completed jobs in the final email notification as ``maxvmsize''.

\emph{Jobs that request a low-memory footprint are more likely to load on a busy cluster.}

\noindent Other essential options are \option{--time}, or \option{-t}, and \option{--account}, or \option{-A}.
\begin{itemize}
    \item \option{--time=<time>} -- is the estimate of wall clock time required for your job to run. 
    As previously mentioned, the maximum is 7 days for batch and 24 hours for interactive jobs. 
    Jobs with a smaller \texttt{time} value will have a higher priority and may result in your job being scheduled sooner. 

    \item \option{--account=<name>} -- specifies which Account, aka project or association, 
    that the Speed resources your job uses should be attributed to. When moving from 
    GE to SLURM users most users were assigned to Speed's two default accounts 
    \texttt{speed1} and \texttt{speed2}. However, users that belong to a particular research
    group or project are will have a default Account like the following
    \texttt{aits},
    \texttt{vidpro},
    \texttt{gipsy},
    \texttt{ai2},
    \texttt{mpackir},
    \texttt{cmos}, among others.
\end{itemize}