% 2.10 SSH Keys for MPI
% -------------------------------------------------------------
\subsection{SSH Keys for MPI}
\label{sect:ssh-mpi}

Some programs, such as Fluent, utilize MPI (Message Passing Interface) for parallel processing.
MPI requires `passwordless login', which is achieved through SSH keys.
Here are the steps to set up SSH keys for MPI:

\begin{itemize}
	\item Navigate to the \texttt{.ssh} directory
	\begin{verbatim}
	cd ~/.ssh
	\end{verbatim}

	\item Generate a new SSH key pair (Accept the default location and leave the passphrase blank)
	\begin{verbatim}
	ssh-keygen -t ed25519
	\end{verbatim}

	\item Authorize the Public Key:
	\begin{verbatim}
	cat id_ed25519.pub >> authorized_keys
	\end{verbatim}
	If the \texttt{\href{https://www.ssh.com/academy/ssh/authorized-keys-file}{authorized\_keys}} file does not exist, use
	\begin{verbatim}
	cat id_ed25519.pub > authorized_keys
	\end{verbatim}

	\item Set permissions: ensure the correct permissions are set for the `authorized\_keys' file and your home directory
	(most users will already have these permissions by default):
	\begin{verbatim}
	chmod 600 ~/.ssh/authorized_keys
	chmod 700 ~
	\end{verbatim}
\end{itemize}