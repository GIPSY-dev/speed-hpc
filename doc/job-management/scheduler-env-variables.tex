% 2.9 Scheduler Environment Variables
% -------------------------------------------------------------
\subsection{Scheduler Environment Variables}
\label{sect:env-vars}

The scheduler provides several environment variables that can be useful in your job scripts.
These variables can be accessed within the job using commands like \tool{env} or \tool{printenv}.
Many of these variables start with the prefix \texttt{SLURM}.

\noindent Here are some of the most useful environment variables:

\begin{itemize}
	\item
	\api{\$TMPDIR} (and \api{\$SLURM\_TMPDIR}):
	% TODO: verify temporal existence
	This is the path to the job's temporary space on the node. It \emph{only} exists for the duration of the job.
	If you need the data from this temporary space, ensure you copy it before the job terminates.

	\item
	\api{\$SLURM\_SUBMIT\_DIR}:
	The path to the job's working directory (likely an NFS-mounted path).
	If, \option{--chdir}, was stipulated, that path is taken; if not,
	the path defaults to your home directory.

	\item
	\api{\$SLURM\_JOBID}:
	This variable holds the current job's ID, which is useful for job
	manipulation and reporting within the job's process.

	\item
	\api{\$SLURM\_NTASKS}: the number of cores requested for the job. This variable can
	be used in place of hardcoded thread-request declarations, e.g., for
	Fluent or similar.

	\item
	\api{\$SLURM\_JOB\_NODELIST}:
	This lists the nodes participating in your job.

	\item \api{\$SLURM\_ARRAY\_TASK\_ID}:
	For array jobs, this variable represents the task ID
	(refer to \xs{sect:array-jobs} for more details on array jobs).
\end{itemize}

\noindent For a more comprehensive list of environment variables,
refer to the SLURM documentation for
\href{https://slurm.schedmd.com/srun.html#SECTION_INPUT-ENVIRONMENT-VARIABLES}{Input Environment Variables} and
\href{https://slurm.schedmd.com/srun.html#SECTION_OUTPUT-ENVIRONMENT-VARIABLES}{Output Environment Variables}.

\noindent An example script that utilizes some of these environment variables
is in \xf{fig:tmpdir.sh}.

\begin{figure}[htpb]
    \lstinputlisting[language=csh,frame=single,basicstyle=\scriptsize\ttfamily]{tmpdir.sh}
    \caption{Source code for \file{tmpdir.sh}}
	\label{fig:tmpdir.sh}
\end{figure}