% 2.11 Creating Virtual Environments
% -------------------------------------------------------------
\subsection{Creating Virtual Environments}
\label{sect:creating-venvs}
\label{sect:environments}
\label{sect:examples-venv}

The following documentation is specific to \textbf{Speed}, other clusters may have their own requirements.
%HPC Facility at the Gina Cody School of Engineering and Computer Science.

Virtual environments are typically created using Conda or Python.
Another option is Singularity (detailed in \xs{sect:singularity-containers}).
These environments are usually created once during an interactive session before
submitting a batch job to the scheduler.

The job script submitted to the scheduler should:
\begin{enumerate}
	\item Activate the virtual environment.
	\item Use the virtual environment.
	\item Deactivate the virtual environment at the end of the job.
\end{enumerate}

%  2.11.1 Anaconda
% -------------------
\subsubsection{Anaconda}
\label{sect:conda-venv}

To create an Anaconda environment, follow these steps:
\begin{enumerate}
	\item Request an interactive session
	\begin{verbatim}
		salloc -p pg --gpus=1
	\end{verbatim}

	\item Load the Anaconda module and create your Anaconda environment in your speed-scratch directory by using
	the \option{--prefix} option (without this o tion, the environment will be created in your home directory by default).
	\begin{verbatim}
		module load anaconda3/2023.03/default
		conda create --prefix /speed-scratch/$USER/myconda
	\end{verbatim}

	\item List environments (to view your conda environment)
	\begin{verbatim}
		conda info --envs
		# conda environments:
		#
		base                  *  /encs/pkg/anaconda3-2023.03/root
                         		 /speed-scratch/a_user/myconda
	\end{verbatim}

	\item Activate the environment
	\begin{verbatim}
		conda activate /speed-scratch/$USER/myconda
	\end{verbatim}

	\item Add \tool{pip} to your environment (this will install \tool{pip} and \tool{pip}'s dependencies,
	including \tool{python}, into the environment.)
	\begin{verbatim}
		conda install pip
	\end{verbatim}
\end{enumerate}

\noindent A consolidated example using Conda:
\begin{verbatim}
    salloc --mem=10G --gpus=1 -p pg -A <slurm account name>
    mkdir -p /speed-scratch/$USER
    cd /speed-scratch/$USER
    module load anaconda3/2023.03/default
    conda create -p /speed-scratch/$USER/pytorch-env
    conda activate /speed-scratch/$USER/pytorch-env
    conda install python=3.11.0
    pip3 install torch torchvision torchaudio --index-url \
    https://download.pytorch.org/whl/cu117
    ....
    conda deactivate
    exit # end the salloc session
\end{verbatim}

\noindent If you encounter \textbf{no space left error} while creating Conda environments, please refer to
\xa{sect:quota-exceeded}.
Likely you forgot \option{--prefix} or environment variables below.

\textbf{Important Note:} \tool{pip} (and \tool{pip3}) are package installers for Python. When you use
\texttt{pip install}, it installs packages from the Python Package Index (PyPI), whereas,
\texttt{conda install} installs packages from Anaconda's repository.

\paragraph{Conda Env without \option{--prefix}}

If you don't want to use the \option{--prefix} option every time you create a new environment and
do not want to use the default home directory, you can create a new directory and set the following
variables to point to the newly created directory, e.g.:
\begin{verbatim}
    mkdir -p /speed-scratch/$USER/conda
    setenv CONDA_ENVS_PATH /speed-scratch/$USER/conda
    setenv CONDA_PKGS_DIRS /speed-scratch/$USER/conda/pkg
\end{verbatim}

\noindent If you want to make these changes permanent, add the variables to your \texttt{.tcshrc}
or \texttt{.bashrc} (depending on the default shell you are using).


% 2.11.2 Python
% -------------------
\subsubsection{Python}
\label{sect:python-venv}

Setting up a Python virtual environment is straightforward.
Here's an example that use a Python virtual environment:

\begin{verbatim}
    salloc --mem=10G --gpus=1 -p pg -A <slurm account name>
    mkdir -p /speed-scratch/$USER
    cd /speed-scratch/$USER
    module load python/3.9.1/default
    mkdir -p /speed-scratch/$USER/tmp
    setenv TMPDIR /speed-scratch/$USER/tmp
    setenv TMP /speed-scratch/$USER/tmp
    python -m venv $TMPDIR/testenv (testenv=name of the virtualEnv)
    source /speed-scratch/$USER/tmp/testenv/bin/activate.csh
    pip install <modules>
    deactivate
    exit
\end{verbatim}

\noindent See, e.g.,
\href{https://github.com/NAG-DevOps/speed-hpc/blob/master/src/gurobi-with-python.sh}
{\texttt{gurobi-with-python.sh}}

\noindent\textbf{Important Note:} our partition \texttt{ps} is used for CPU jobs, while \texttt{pg},
\texttt{pt}, and \texttt{cl} are used for GPU jobs. You do not need to use \option{--gpus}
when preparing environments for CPU jobs.

\noindent\textbf{Note:} Python enviornments are also preferred over Conda
in some clusters, see a note in~\xs{sect:jupyter-python}.