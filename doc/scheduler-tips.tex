% ------------------------------------------------------------------------------
\subsection{Tips/Tricks}
\label{sect:tips}

\begin{itemize}
\item
Files/scripts must have Linux line breaks in them (not Windows ones).
Use \tool{file} command to verify; and \tool{dos2unix} command
to convert.

\item
Use \tool{rsync}, not \tool{scp}, when moving a lot of data around.

\item
If you are going to move many many files between NFS-mounted storage and the 
cluster, \tool{tar} everything up first. 

\item
If you intend to use a different shell (e.g., \tool{bash}~\cite{aosa-book-vol1-bash}),
%you will need to source a different scheduler file, and
you will need to change the shell declaration in your script(s).

% TODO:
%\item
%The load displayed in \tool{qstat} by default is \api{np\_load}, which is
%load/\#cores. That means that a load of, ``1'', which represents a fully active 
%core, is displayed as $0.03$ on the node in question, as there are 32 cores 
%on a node. To display load ``as is'' (such that a node with a fully active 
%core displays a load of approximately $1.00$), add the following to your
%\file{.tcshrc} file: \texttt{setenv SGE\_LOAD\_AVG load\_avg}

\item
\textbf{Try to request resources that closely match what your job will use: 
requesting many more cores or much more memory than will be needed makes a 
job more difficult to schedule when resources are scarce.}

\item
E-mail, \texttt{rt-ex-hpc AT encs.concordia.ca}, with any concerns/questions.
\end{itemize}
