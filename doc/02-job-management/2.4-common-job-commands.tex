% 2.4 Common Job Management Commands
% -------------------------------------------------------------
\subsection{Common Job Management Commands}
\label{sect:job-management-commands}

Here is a summary of useful job management commands for handling various aspects of
job submission and monitoring on the Speed cluster:

\begin{itemize}
    \item Submitting a job:
	\begin{verbatim}
		sbatch -A <ACCOUNT> --mem=<MEMORY> -p <PARTITION> ./<myscript>.sh
	\end{verbatim}

	\item Checking your job(s) status:
	\begin{verbatim}
		squeue -u <ENCSusername>
	\end{verbatim}

	\item Displaying cluster status:
	\begin{verbatim}
		squeue
	\end{verbatim}
		\begin{itemize}
			\item Use \option{-A} for per account (e.g., \texttt{-A vidpro}, \texttt{-A aits})
			\item Use \option{-p} for per partition (e.g., \texttt{-p ps}, \texttt{-p pg}, \texttt{-p pt}), etc.
		\end{itemize}

	\item Displaying job information:
	\begin{verbatim}
		squeue --job <job-ID>
	\end{verbatim}

	\item Displaying individual job steps: (to see which step failed if you used \tool{srun})
	\begin{verbatim}
		squeue -las
	\end{verbatim}

	\item Monitoring job and cluster status: (view \tool{sinfo} and watch the queue for your job(s))
	\begin{verbatim}
		watch -n 1 "sinfo -Nel -pps,pt,pg,pa && squeue -la"
	\end{verbatim}

	\item Canceling a job:
	\begin{verbatim}
		scancel <job-ID>
	\end{verbatim}

	\item Holding a job:
	\begin{verbatim}
		scontrol hold <job-ID>
	\end{verbatim}

	\item Releasing a job:
	\begin{verbatim}
		scontrol release <job-ID>
	\end{verbatim}

	\item Getting job statistics: (including useful metrics like ``maxvmem'')
	\begin{verbatim}
		sacct -j <job-ID>
	\end{verbatim}
    \api{maxvmem} is one of the more useful stats that you can elect to display as a format option.

    \small
	\begin{verbatim}
	% sacct -j 2654
	JobID           JobName  Partition    Account  AllocCPUS      State ExitCode
	------------ ---------- ---------- ---------- ---------- ---------- --------
	2654               envs         ps     speed1          1  COMPLETED      0:0
	2654.batch        batch                speed1          1  COMPLETED      0:0
	2654.extern      extern                speed1          1  COMPLETED      0:0
	% sacct -j 2654 -o jobid,user,account,MaxVMSize,Reason%10,TRESUsageOutMax%30
	JobID             User    Account  MaxVMSize     Reason        TRESUsageOutMax
	------------ --------- ---------- ---------- ---------- ----------------------
	2654           serguei     speed1                  None
	2654.batch                 speed1    296840K             energy=0,fs/disk=1975
	2654.extern                speed1    296312K              energy=0,fs/disk=343
	\end{verbatim}
    \normalsize

	See \texttt{man sacct} or \texttt{sacct -e} for details of the available formatting options.
	You can define your preferred default format in the \api{SACCT\_FORMAT} environment variable
	in your \texttt{.cshrc} or \texttt{.bashrc} files.

	\item Displaying job efficiency: (including CPU and memory utilization)
	\begin{verbatim}
	    seff <job-ID>
	\end{verbatim}
    Don't execute it on \texttt{RUNNING} jobs (only on completed/finished jobs), else
	efficiency statistics may be misleading. If you define the following directive in your batch script,
    your GCS ENCS email address will receive an email with \tool{seff}'s output when your job is finished.
	\begin{verbatim}
	    #SBATCH --mail-type=ALL
	\end{verbatim}

	Output example:
	\begin{verbatim}
    Job ID: XXXXX
    Cluster: speed
    User/Group: user1/user1
    State: COMPLETED (exit code 0)
    Nodes: 1
    Cores per node: 4
    CPU Utilized: 00:04:29
    CPU Efficiency: 0.35% of 21:32:20 core-walltime
    Job Wall-clock time: 05:23:05
    Memory Utilized: 2.90 GB
    Memory Efficiency: 2.90% of 100.00 GB
	\end{verbatim}
\end{itemize}