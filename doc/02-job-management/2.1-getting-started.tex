% 2.1 Getting Started
% -------------------------------------------------------------
\subsection{Getting Started}
\label{sect:getting-started}

Before getting started, please review the ``What Speed is'' (\xs{sect:speed-is-for})
and ``What Speed is Not'' (\xs{sect:speed-is-not}).
Once your GCS ENCS account has been granted access to ``Speed'',
use your GCS ENCS account credentials to create an SSH connection to
\texttt{speed} (an alias for \texttt{speed-submit.encs.concordia.ca}).

All users are expected to have a basic understanding of
Linux and its commonly used commands (see \xa{sect:faqs} for resources).

%  2.1.1 SSH Connection
% -----------------------
\subsubsection{SSH Connections}
\label{sect:ssh-connection}

Requirements to create SSH connection to ``Speed'':
\begin{enumerate}
	\item \textbf{Active GCS ENCS user account:} Ensure you have an active GCS ENCS user account with
	permission to connect to Speed (see \xs{sect:access-requests}).
	\item \textbf{VPN Connection} (for off-campus access): If you are off-campus, you wil need to establish an active connection to Concordia's VPN,
	which requires a Concordia netname.
	\item \textbf{Terminal Emulator for Windows:} Windows systems use a terminal emulator such as PuTTY, Cygwin, or MobaXterm.
	\item \textbf{Terminal for macOS:} macOS systems have a built-in Terminal app or \tool{xterm} that comes with XQuartz.
\end{enumerate}

\noindent To create an SSH connection to Speed, open a terminal window and type the following command, replacing \verb!<ENCSusername>! with your ENCS account's username:
\begin{verbatim}
    ssh <ENCSusername>@speed.encs.concordia.ca
\end{verbatim}

\noindent For detailed instructions on securely connecting to a GCS server, refer to the AITS FAQ:
\href{https://www.concordia.ca/ginacody/aits/support/faq/ssh-to-gcs.html}{How do I securely connect to a GCS server?}

%  2.1.2 Environment Set Up
% --------------------------
\subsubsection{Environment Set Up}
\label{sect:envsetup}
%TO BE DELETED
%% ------------------------------------------------------------------------------
\subsubsection{Environment Set Up}
\label{sect:envsetup}

After creating an SSH connection to ``Speed'', you will need to
make sure the \tool{srun}, \tool{sbatch}, and \tool{salloc}
commands are available to you. 
Type the command name at the linux prompt and press enter.
If the command is not available, e.g.,  (``command not found'') is returned,
you need to make sure your \api{\$PATH} has \texttt{/local/bin} in it.
To view your \api{\$PATH} type \texttt{echo \$PATH} at the linux prompt.
%
%source 
%the ``Altair Grid Engine (AGE)'' scheduler's settings file. 
%Sourcing the settings file will set the environment variables required to 
%execute scheduler commands.
%
%Based on the UNIX shell type, choose one of the following commands to source
%the settings file. 
%
%csh/\tool{tcsh}:
%\begin{verbatim}
%source /local/pkg/uge-8.6.3/root/default/common/settings.csh 
%\end{verbatim}
%
%Bourne shell/\tool{bash}:
%\begin{verbatim}
%. /local/pkg/uge-8.6.3/root/default/common/settings.sh 
%\end{verbatim}
%
%In order to set up the default ENCS bash shell, executing the following command 
%is also required:
%\begin{verbatim}
%printenv ORGANIZATION | grep -qw ENCS || . /encs/Share/bash/profile 
%\end{verbatim}
%
%To verify that you have access to the scheduler commands execute 
%\texttt{qstat -f -u "*"}. If an error is returned, attempt sourcing 
%the settings file again.

The next step is to copy a job template to your home directory and to set up your
cluster-specific storage. Execute the following command from within your
home directory. (To move to your home directory, type \texttt{cd} at the Linux
prompt and press \texttt{Enter}.) 

\begin{verbatim}
cp /home/n/nul-uge/template.sh . && mkdir /speed-scratch/$USER
\end{verbatim}

%\textbf{Tip:} Add the source command to your shell-startup script. 

\textbf{Tip:} the default shell for GCS ENCS users is \tool{tcsh}.
If you would like to use \tool{bash}, please contact 
\texttt{rt-ex-hpc AT encs.concordia.ca}.

%For \textbf{new GCS ENCS Users}, and/or those who don't have a shell-startup script, 
%based on your shell type use one of the following commands to copy a start up script 
%from \texttt{nul-uge}'s home directory to your home directory. (To move to your home
%directory, type \tool{cd} at the Linux prompt and press \texttt{Enter}.)

%csh/\tool{tcsh}:
%\begin{verbatim}
%cp /home/n/nul-uge/.tcshrc . 
%\end{verbatim}

%Bourne shell/\tool{bash}:
%\begin{verbatim}
%cp /home/n/nul-uge/.bashrc . 
%\end{verbatim}

%Users who already have a shell-startup script, can use a text editor, such as
%\tool{vim} or \tool{emacs}, to add the source request to your existing
%shell-startup environment (i.e., to the \file{.tcshrc} file in your home directory). 

%csh/\tool{tcsh}:
%Sample \file{.tcshrc} file:
%\begin{verbatim}
%# Speed environment set up 
%if ($HOSTNAME == speed-submit.encs.concordia.ca) then
   %source /local/pkg/uge-8.6.3/root/default/common/settings.csh
%endif
%\end{verbatim}
%
%Bourne shell/\tool{bash}:
%Sample \file{.bashrc} file:
%\begin{verbatim}
%# Speed environment set up 
%if [ $HOSTNAME = "speed-submit.encs.concordia.ca" ]; then
    %. /local/pkg/uge-8.6.3/root/default/common/settings.sh
    %printenv ORGANIZATION | grep -qw ENCS || . /encs/Share/bash/profile
%fi
%\end{verbatim}

%\noindent
%\textbf{NOTE:} If you have used UGE commands in the past you probably still have these
%lines there; \textbf{they should now be removed}, as they have no use in SLURM:

%csh/\tool{tcsh}:
%Sample \file{.tcshrc} file:
%\begin{verbatim}
%# Speed environment set up 
%if ($HOSTNAME == speed-submit.encs.concordia.ca) then
%   source /local/pkg/uge-8.6.3/root/default/common/settings.csh
%endif
%\end{verbatim}

%Bourne shell/\tool{bash}:
%Sample \file{.bashrc} file:
%\begin{verbatim}
%# Speed environment set up 
%if [ $HOSTNAME = "speed-submit.encs.concordia.ca" ]; then
%    . /local/pkg/uge-8.6.3/root/default/common/settings.sh
%    printenv ORGANIZATION | grep -qw ENCS || . /encs/Share/bash/profile
%fi
%\end{verbatim}

%Note that you will need to either log out and back in, or execute a new shell, 
%for the environment changes in the updated \file{.tcshrc} or \file{.bashrc} file to be applied 
%(\textbf{important}).

%

After creating an SSH connection to Speed, you will need to make sure the \tool{srun}, \tool{sbatch}, and \tool{salloc}
commands are available to you. To check this, type each command at the prompt and press Enter.
If ``command not found'' is returned, you need to make sure your \api{\$PATH} includes \texttt{/local/bin}.
You can check your path by typing:
\begin{verbatim}
    echo $PATH
\end{verbatim}

\noindent The next step is to set up your cluster-specific storage ``speed-scratch'', to do so, execute the following command from within your
home directory.
\begin{verbatim}
    mkdir -p /speed-scratch/$USER && cd /speed-scratch/$USER
\end{verbatim}

\noindent Next, copy a job template to your cluster-specific storage
\begin{itemize}
    \item From Windows drive G: to Speed:\\
    \verb|cp /winhome/<1st letter of $USER>/$USER/<script>.sh /speed-scratch/$USER/|
    \item From Linux drive U: to Speed:\\
    \verb|cp ~/<script>.sh /speed-scratch/$USER/|
\end{itemize}

\noindent \textbf{Tip:} the default shell for GCS ENCS users is \tool{tcsh}.
If you would like to use \tool{bash}, please contact \texttt{rt-ex-hpc AT encs.concordia.ca}.

\noindent \textbf{Note:} If you encounter a ``command not found'' error after logging in to Speed,
your user account may have defunct Grid Engine environment commands.
See \xa{appdx:uge-to-slurm} for instructions on how to resolve this issue.