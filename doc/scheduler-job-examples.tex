% ------------------------------------------------------------------------------
\subsection{Example Job Script: Fluent}

\begin{figure}[htpb]
    \lstinputlisting[language=csh,frame=single,basicstyle=\footnotesize\ttfamily]{fluent.sh}
    \caption{Source code for \file{fluent.sh}}
	\label{fig:fluent.sh}
\end{figure}

The job script in \xf{fig:fluent.sh} runs Fluent in parallel over 32 cores. 
%Of note, we have requested e-mail notifications (\texttt{-m}), are defining the 
Of note, we have requested e-mail notifications (\texttt{--mail-type}), are defining the 
%parallel environment for, \tool{fluent}, with, \texttt{-sgepe smp} (\textbf{very 
parallel environment for, \tool{fluent}, with, \texttt{-t\$SLURM\_NTASKS} and \texttt{-g-cnf=\$FLUENTNODES} (\textbf{very 
important}), and are setting \api{\$TMPDIR} as the in-job location for the
``moment'' \file{rfile.out} file (in-job, because the last line of the script 
copies everything from \api{\$TMPDIR} to a directory in the user's NFS-mounted home). 
Job progress can be monitored by examining the standard-out file (e.g.,
%\file{flu10000.o249}), and/or by examining the ``moment'' file in 
\texttt{slurm-249.out}), and/or by examining the ``moment'' file in 
\texttt{/disk/nobackup/<yourjob>} (hint: it starts with your job-ID) on the node running
the job. \textbf{Caveat:} take care with journal-file file paths.

% ------------------------------------------------------------------------------
\subsection{Example Job: efficientdet}

The following steps describing how to create an efficientdet environment on
\emph{Speed}, were submitted by a member of Dr. Amer's research group.

\begin{itemize}
    \item 
    Enter your ENCS user account's speed-scratch directory\\ 
    \verb!cd /speed-scratch/<encs_username>!
    \item
		Next
		\begin{itemize}
			\item 
    load python \verb!module load python/3.8.3!
			\item 
    create virtual environment \verb!python3 -m venv <env_name>!
			\item 
    activate virtual environment \verb!source <env_name>/bin/activate.csh!
			\item 
    install DL packages for Efficientdet
		\end{itemize}
\end{itemize}
\small
\begin{verbatim}
pip install tensorflow==2.7.0
pip install lxml>=4.6.1
pip install absl-py>=0.10.0
pip install matplotlib>=3.0.3
pip install numpy>=1.19.4
pip install Pillow>=6.0.0
pip install PyYAML>=5.1
pip install six>=1.15.0
pip install tensorflow-addons>=0.12
pip install tensorflow-hub>=0.11
pip install neural-structured-learning>=1.3.1
pip install tensorflow-model-optimization>=0.5
pip install Cython>=0.29.13
pip install git+https://github.com/cocodataset/cocoapi.git#subdirectory=PythonAPI
\end{verbatim}
\normalsize

% ------------------------------------------------------------------------------
\subsection{Java Jobs}
\label{sect:java}

Jobs that call \tool{java} have a memory overhead, which needs to be taken 
%into account when assigning a value to \api{h\_vmem}. Even the most basic 
into account when assigning a value to \option{--mem}. Even the most basic 
\tool{java} call, \texttt{java -Xmx1G -version}, will need to have,
%\texttt{-l h\_vmem=5G}, with the 4-GB difference representing the memory overhead. 
\texttt{--mem=5G}, with the 4-GB difference representing the memory overhead. 
Note that this memory overhead grows proportionally with the value of
\texttt{-Xmx}. To give you an idea, when \texttt{-Xmx} has a value of 100G,
%\api{h\_vmem} has to be at least 106G; for 200G, at least 211G; for 300G, at least 314G.
\option{--mem} has to be at least 106G; for 200G, at least 211G; for 300G, at least 314G.

% TODO: add MARF and GIPSY Java jobs

% ------------------------------------------------------------------------------
\subsection{Scheduling On The GPU Nodes}

The primary cluster has two GPU nodes, each with six Tesla (CUDA-compatible) P6
cards: each card has 2048 cores and 16GB of RAM. Though note that the P6
is mainly a single-precision card, so unless you need the GPU double
precision, double-precision calculations will be faster on a CPU node.

Job scripts for the GPU queue differ in that they
%do not need these
%statements:
%
%\begin{verbatim}
%#$ -pe smp <threadcount>
%#$ -l h_vmem=<memory>G
%\end{verbatim}
%
%But do
need this statement, which attaches either a single GPU, or, two
GPUs, to the job:

%\begin{verbatim}
%#$ -l gpu=[1|2]
%\end{verbatim}
\begin{verbatim}
#SBATCH --gpus=[1|2]
\end{verbatim}

% TODO: verify accuracy
% Single-GPU jobs are granted 5~CPU cores and 80GB of system memory, and
% dual-GPU jobs are granted 10~CPU cores and 160GB of system memory. A
% total of \emph{four} GPUs can be actively attached to any one user at any given
% time.

Once that your job script is ready, you can submit it to the GPU partition (queue)
with:

%\begin{verbatim}
%qsub -q g.q ./<myscript>.sh
%\end{verbatim}
\begin{verbatim}
sbatch -p pg ./<myscript>.sh
\end{verbatim}

And you can query \tool{nvidia-smi} on the node that is running your job with:

\begin{verbatim}
ssh <username>@speed[-05|-17|37-43] nvidia-smi
\end{verbatim}

Status of the GPU queue can be queried with:

%\begin{verbatim}
%qstat -f -u "*" -q g.q
%\end{verbatim}
\begin{verbatim}
sinfo -p pg --long --Node
\end{verbatim}

\noindent
\textbf{Very important note} regarding TensorFlow and PyTorch: 
if you are planning to run TensorFlow and/or PyTorch multi-GPU jobs, 
\textbf{do not} use the \api{tf.distribute} and/or\\
\api{torch.nn.DataParallel} 
functions on \textbf{speed-01,05,17}, as they will crash the compute node (100\% certainty). 
This appears to be the current hardware's architecture's defect.
%
The workaround is to either
% TODO: Need to link to that example
manually effect GPU parallelisation (TensorFlow has an example on how to
do this), or to run on a single GPU.

\vspace{10pt}
\noindent
\textbf{Important}
\vspace{10pt}

%Users without permission to use the GPU nodes can submit jobs to the \texttt{g.q}
Users without permission to use the GPU nodes can submit jobs to the \texttt{pg}
partition, but those jobs will hang and never run.
%
%There are two GPUs in both \texttt{speed-05} and \texttt{speed-17}, and one 
%in \texttt{speed-19}.
Their availability is seen with:
%, \texttt{qstat -F g} (note the capital): 
%
%\small
%\begin{verbatim}
%queuename                      qtype resv/used/tot. load_avg arch          states
%---------------------------------------------------------------------------------
%...
%---------------------------------------------------------------------------------
%g.q@speed-05.encs.concordia.ca BIP   0/0/32         0.04     lx-amd64
        %hc:gpu=6
%---------------------------------------------------------------------------------
%g.q@speed-17.encs.concordia.ca BIP   0/0/32         0.01     lx-amd64
        %hc:gpu=6
%---------------------------------------------------------------------------------
%...
%---------------------------------------------------------------------------------
%s.q@speed-19.encs.concordia.ca BIP   0/32/32        32.37    lx-amd64
        %hc:gpu=1
%---------------------------------------------------------------------------------
%etc. 
%\end{verbatim}
%\normalsize

\scriptsize
\begin{verbatim}
[serguei@speed-submit src] % sinfo -p pg --long --Node
Thu Oct 19 22:31:04 2023
NODELIST   NODES PARTITION       STATE CPUS    S:C:T MEMORY TMP_DISK WEIGHT AVAIL_FE REASON
speed-05       1        pg        idle 32     2:16:1 515490        0      1    gpu16 none
speed-17       1        pg     drained 32     2:16:1 515490        0      1    gpu16 UGE
speed-25       1        pg        idle 32     2:16:1 257458        0      1    gpu32 none
speed-27       1        pg        idle 32     2:16:1 257458        0      1    gpu32 none
[serguei@speed-submit src] % sinfo -p pt --long --Node
Thu Oct 19 22:32:39 2023
NODELIST   NODES PARTITION       STATE CPUS    S:C:T MEMORY TMP_DISK WEIGHT AVAIL_FE REASON
speed-37       1        pt        idle 256    2:64:2 980275        0      1 gpu20,mi none
speed-38       1        pt        idle 256    2:64:2 980275        0      1 gpu20,mi none
speed-39       1        pt        idle 256    2:64:2 980275        0      1 gpu20,mi none
speed-40       1        pt        idle 256    2:64:2 980275        0      1 gpu20,mi none
speed-41       1        pt        idle 256    2:64:2 980275        0      1 gpu20,mi none
speed-42       1        pt        idle 256    2:64:2 980275        0      1 gpu20,mi none
speed-43       1        pt        idle 256    2:64:2 980275        0      1 gpu20,mi none
\end{verbatim}

This status demonstrates that most are available (i.e., have not been 
requested as resources). To specifically request a GPU node, add,
%\texttt{-l g=[\#GPUs]}, to your \tool{qsub} (statement/script) or
\texttt{--gpus=[\#GPUs]}, to your \tool{sbatch} (statement/script) or
%\tool{qlogin} (statement) request. For example,
\tool{salloc} (statement) request. For example,
%\texttt{qsub -l h\_vmem=1G -l g=1 ./count.sh}. You 
\texttt{sbatch -t 10 --mem=1G --gpus=1 -p pg ./tcsh.sh}. You 
will see that this job has been assigned to one of the GPU nodes.

%\small
%\begin{verbatim}
%queuename                      qtype resv/used/tot. load_avg arch          states
%--------------------------------------------------------------------------------- 
%g.q@speed-05.encs.concordia.ca BIP 0/0/32 0.01 lx-amd64  hc:gpu=6 
%--------------------------------------------------------------------------------- 
%g.q@speed-17.encs.concordia.ca BIP 0/0/32 0.01 lx-amd64  hc:gpu=6 
%--------------------------------------------------------------------------------- 
%s.q@speed-19.encs.concordia.ca BIP 0/1/32 0.04 lx-amd64  hc:gpu=0 (haff=1.000000) 
       %538 100.00000 count.sh   sbunnell     r     03/07/2019 02:39:39     1
%---------------------------------------------------------------------------------
%etc. 
%\end{verbatim}
%\normalsize

%\small
%\begin{verbatim}
%queuename                      qtype resv/used/tot. load_avg arch          states
%--------------------------------------------------------------------------------- 
%g.q@speed-05.encs.concordia.ca BIP 0/0/32 0.01 lx-amd64  hc:gpu=6 
%--------------------------------------------------------------------------------- 
%g.q@speed-17.encs.concordia.ca BIP 0/0/32 0.01 lx-amd64  hc:gpu=6 
%--------------------------------------------------------------------------------- 
%s.q@speed-19.encs.concordia.ca BIP 0/1/32 0.04 lx-amd64  hc:gpu=0 (haff=1.000000) 
       %538 100.00000 count.sh   sbunnell     r     03/07/2019 02:39:39     1
%---------------------------------------------------------------------------------
%etc. 
%\end{verbatim}
%\normalsize

\scriptsize
\begin{verbatim}
[serguei@speed-submit src] % squeue -p pg -o "%15N %.6D %7P %.11T %.4c %.8z %.6m %.8d %.6w %.8f %20G %20E"
NODELIST         NODES PARTITI       STATE MIN_    S:C:T MIN_ME MIN_TMP_  WCKEY FEATURES GROUP DEPENDENCY
speed-05             1 pg          RUNNING    1    *:*:*     1G        0 (null)   (null) 11929     (null)
[serguei@speed-submit src] % sinfo -p pg -o "%15N %.6D %7P %.11T %.4c %.8z %.6m %.8d %.6w %.8f %20G %20E"
NODELIST         NODES PARTITI       STATE CPUS    S:C:T MEMORY TMP_DISK WEIGHT AVAIL_FE GRES      REASON
speed-17             1 pg          drained   32   2:16:1 515490        0      1    gpu16 gpu:6        UGE
speed-05             1 pg            mixed   32   2:16:1 515490        0      1    gpu16 gpu:6       none
speed-[25,27]        2 pg             idle   32   2:16:1 257458        0      1    gpu32 gpu:2       none
\end{verbatim}
\normalsize

%And that there are no more GPUs available on that node (\texttt{hc:gpu=0}).
%Note that no more than two GPUs can be requested for any one job. 

% ------------------------------------------------------------------------------
\subsubsection{P6 on Multi-GPU, Multi-Node}

As described lines above, P6 cards are not compatible with Distribute and DataParallel functions
(\texttt{Pytorch, Tensorflow}) when running on Multi-GPUs.
One workaround is to run the job in Multi-node, single GPU per node; per example:

\begin{verbatim}
#SBATCH --nodes=2
#SBATCH --gpus-per-node=1
\end{verbatim}

On P6 nodes: \texttt{speed-05, speed-17, speed-01}

The example: 
  \href{https://github.com/NAG-DevOps/speed-hpc/blob/master/src/pytorch-multinode-multigpu.sh}
  {pytorch-multinode-multigpu.sh}
illustrates a job for training on Multi-nodes, Multi-GPUs

% ------------------------------------------------------------------------------
\subsubsection{CUDA}

When calling \tool{CUDA} within job scripts, it is important to create a link to
the desired \tool{CUDA} libraries and set the runtime link path to the same libraries. 
For example, to use the \texttt{cuda-11.5} libraries, specify the following in 
your \texttt{Makefile}.

\begin{verbatim}
-L/encs/pkg/cuda-11.5/root/lib64 -Wl,-rpath,/encs/pkg/cuda-11.5/root/lib64
\end{verbatim}

In your job script, specify the version of \texttt{gcc} to use prior to calling 
cuda. For example: 
   \texttt{module load gcc/8.4}
or
   \texttt{module load gcc/9.3}

% ------------------------------------------------------------------------------
\subsubsection{Special Notes for sending CUDA jobs to the GPU Queue}

%It is not possible to create a \texttt{qlogin} session on to a node in the 
%\textbf{GPU Queue} (\texttt{g.q}). As direct logins to these nodes is not 
%available,
Interactive 
jobs (\xs{sect:interactive-jobs}) must be submitted to the \textbf{GPU partition} in order to compile 
and link.
%
We have several versions of CUDA installed in:
\begin{verbatim}
/encs/pkg/cuda-11.5/root/
/encs/pkg/cuda-10.2/root/
/encs/pkg/cuda-9.2/root
\end{verbatim}

For CUDA to compile properly for the GPU partition, edit your \texttt{Makefile}
replacing \option{\/usr\/local\/cuda} with one of the above.

% ------------------------------------------------------------------------------
\subsubsection{OpenISS Examples}
\label{sect:openiss-examples}

These represent more comprehensive research-like examples of
jobs for computer vision and other tasks with a lot longer
runtime (a subject to the number of epochs and other parameters)
derive from the actual research works of students and their
theses.
%
These jobs require the use of CUDA and GPUs.
These examples are available as ``native'' jobs on Speed
and as Singularity containers.

% ------------------------------------------------------------------------------
\paragraph{OpenISS and REID}
\label{sect:openiss-reid}

The example
  \href{https://github.com/NAG-DevOps/speed-hpc/blob/master/src/openiss-reid-speed.sh}
  {openiss-reid-speed.sh}
illustrates a job for a computer-vision based person re-identification
(e.g., motion capture-based tracking for stage performance) part of the OpenISS
project by Haotao Lai~\cite{lai-haotao-mcthesis19} using TensorFlow and Keras.
The fork of the original repo~\cite{openiss-reid-tfk} adjusted to to run on Speed is here:

\begin{itemize}
	\item \url{https://github.com/NAG-DevOps/openiss-reid-tfk}
\end{itemize}

and its detailed description on how to run it on Speed is
in the README:

\begin{itemize}
	\item \url{https://github.com/NAG-DevOps/speed-hpc/tree/master/src#openiss-reid-tfk}
\end{itemize}

% ------------------------------------------------------------------------------
\paragraph{OpenISS and YOLOv3}
\label{sect:openiss-yolov3}

The related code using YOLOv3 framework is in the
the fork of the original repo~\cite{openiss-yolov3} adjusted
to to run on Speed is here:

\begin{itemize}
	\item \url{https://github.com/NAG-DevOps/openiss-yolov3}
\end{itemize}

Its example job scripts can run on both CPUs and GPUs,
as well as interactively using TensorFlow:

\begin{itemize}
	\item Interactive mode:
  \href{https://github.com/NAG-DevOps/speed-hpc/blob/master/src/openiss-yolo-interactive.sh}
  {openiss-yolo-interactive.sh}
	\item CPU-based job:
  \href{https://github.com/NAG-DevOps/speed-hpc/blob/master/src/openiss-yolo-cpu.sh}
  {openiss-yolo-cpu.sh}
	\item GPU-based jon:
  \href{https://github.com/NAG-DevOps/speed-hpc/blob/master/src/openiss-yolo-gpu.sh}
  {openiss-yolo-gpu.sh}
\end{itemize}

The detailed description on how to run these on Speed is
in the README at:

\begin{itemize}
	\item \url{https://github.com/NAG-DevOps/speed-hpc/tree/master/src#openiss-yolov3}
\end{itemize}

% ------------------------------------------------------------------------------
\subsection{Singularity Containers}
\label{sect:singularity-containers}

If the \texttt{/encs} software tree does not have a required software
instantaneously available, another option is to run Singularity
containers. We run EL7 flavor of Linux, and if some projects
require Ubuntu or other distributions, there is a possibility
to run that software as a container, including the ones
translated from Docker.

The example
  \href{https://github.com/NAG-DevOps/speed-hpc/blob/master/src/lambdal-singularity.sh}
  {lambdal-singularity.sh}
showcases an immediate use of a container built for the Ubuntu-based
LambdaLabs software stack, originally built as a Docker image then
pulled in as a Singularity container that is immediately available
for use as that job example illustrates. The source material
used for the docker image was our fork of their official
repo: \url{https://github.com/NAG-DevOps/lambda-stack-dockerfiles}

NOTE: It is important if you make your own containers or pull from
DockerHub, use your \verb+/speed-scratch/$USER+ directory as these
images may easily consume gigs of space in your home directory
and you'd run out of quota there very fast.

%\noindent
TIP: To check for your quota, and the corresponding commands
to find big files, see: \url{https://www.concordia.ca/ginacody/aits/encs-data-storage.html}

We likewise built equivalent OpenISS (\xs{sect:openiss-examples})
containers from their Docker counter parts as they were used for teaching and
research~\cite{oi-containers-poster-siggraph2023}. The
images from \url{https://github.com/NAG-DevOps/openiss-dockerfiles}
and their DockerHub equivalents \url{https://hub.docker.com/u/openiss}
are found in the same public directory on \verb+/speed-scratch/nag-public+
as the LambdaLabs Singularity image. They all have \texttt{.sif} extension.
Some of them can be ran in both batch or interactive mode, some
make more sense to run interactively. They cover some basics with CUDA,
OpenGL rendering, and computer vision tasks as examples from the OpenISS
library and other libraries, including the base images that use different
distros. We also include Jupyter notebook example with Conda support.

\begin{verbatim}
/speed-scratch/nag-public:

openiss-cuda-conda-jupyter.sif
openiss-cuda-devicequery.sif
openiss-opengl-base.sif
openiss-opengl-cubes.sif
openiss-opengl-triangle.sif
openiss-reid.sif
openiss-xeyes.sif
\end{verbatim}

The currently recommended version of Singularity is
\texttt{singularity/3.10.4/default}.

This section comprises an introduction to working with Singularity, its 
containers, and what can and cannot be done with Singularity on the ENCS 
infrastructure. It is not intended to be an exhaustive presentation of 
Singularity: the program's authors do a good job of that here:
\url{https://www.sylabs.io/docs/}. It also assumes that you have successfully installed 
Singularity on a user-managed/personal system (see next paragraph as to why).

Singularity containers are essentially either built from an existing 
container, or are built from scratch. Building from scratch requires a recipe 
file (think of like a Dockerfile), and the operation \emph{must} be effected as root.
You will not have root on the ENCS infrastructure, so any built-from-scratch containers must be created 
on a user-managed/personal system. Root-level permissions are also required
(in some cases, essential; in others, for proper build functionality) for 
building from an existing container.
%, with one exception (see, Building A 
%Container From An Existing Container).
%
Three types of Singularity containers can be built: file-system; sandbox; 
squashfs. The first two are ``writable'' (meaning that changes can persist 
after the Singularity session ends). File-system containers are built around 
the ext3 file system, and are a read-write ``file'', sandbox containers are 
essentially a directory in an existing read-write space, and squashfs 
containers are a read-only compressed ``file''. Note that file-system 
containers \emph{cannot} be resized once built.

Note that the default build is a squashfs one. Also note what Singularity's 
authors have to say about the builds, ``A common workflow is to use the 
``sandbox'' mode for development of the container, and then build it as a 
default (squashfs) Singularity image when done.'' File-system containers are 
considered to be, ``legacy'', at this point in time. When built, a \emph{very small}
overhead is allotted to a file-system container (think, MB), and that 
\emph{cannot} be changed.

Probably for the most of your workflows you might find there is a
Docker container exists for your tasks, in this case you
can use the docker pull function of Singularity as a part
of you virtual environment setup as an interactive job
allocation:

\small
\begin{verbatim}
salloc --gpus=1 -n8 --mem=4Gb -t60
cd /speed-scratch/$USER/
singularity pull openiss-cuda-devicequery.sif docker://openiss/openiss-cuda-devicequery
INFO:    Converting OCI blobs to SIF format
INFO:    Starting build...
\end{verbatim}
\normalsize

\noindent
This method can be used for converting Docker containers directly on Speed.
On GPU nodes make sure to pass on the \option{--nv} flag to Singularity,
so its containers could access the GPUs. See the linked example.
