% ------------------------------------------------------------------------------
\subsubsection{Environment Set Up}
\label{sect:envsetup}

After creating an SSH connection to ``Speed'', you will need to source 
the ``Altair Grid Engine (AGE)'' scheduler's settings file. 
Sourcing the settings file will set the environment variables required to 
execute scheduler commands.

Based on the UNIX shell type, choose one of the following commands to source
the settings file. 

csh/\tool{tcsh}:
\begin{verbatim}
source /local/pkg/uge-8.6.3/root/default/common/settings.csh 
\end{verbatim}

Bourne shell/\tool{bash}:
\begin{verbatim}
. /local/pkg/uge-8.6.3/root/default/common/settings.sh 
\end{verbatim}

In order to set up the default ENCS bash shell, executing the following command 
is also required:
\begin{verbatim}
printenv ORGANIZATION | grep -qw ENCS || . /encs/Share/bash/profile 
\end{verbatim}

To verify that you have access to the scheduler commands execute 
\texttt{qstat -f -u "*"}. If an error is returned, attempt sourcing 
the settings file again.

The next step is to copy a job template to your home directory and to set up your
cluster-specific storage. Execute the following command from within your
home directory. (To move to your home directory, type \texttt{cd} at the Linux
prompt and press \texttt{Enter}.) 

\begin{verbatim}
cp /home/n/nul-uge/template.sh . && mkdir /speed-scratch/$USER
\end{verbatim}

\textbf{Tip:} Add the source command to your shell-startup script. 

\textbf{Tip:} the default shell for GCS ENCS users is \tool{tcsh}.
If you would like to use \tool{bash}, please contact 
\texttt{rt-ex-hpc AT encs.concordia.ca}.

For \textbf{new ENCS Users}, and/or those who don't have a shell-startup script, 
based on your shell type use one of the following commands to copy a start up script 
from \texttt{nul-uge}'s. home directory to your home directory. (To move to your home
directory, type \tool{cd} at the Linux prompt and press \texttt{Enter}.)

csh/\tool{tcsh}:
\begin{verbatim}
cp /home/n/nul-uge/.tcshrc . 
\end{verbatim}

Bourne shell/\tool{bash}:
\begin{verbatim}
cp /home/n/nul-uge/.bashrc . 
\end{verbatim}

Users who already have a shell-startup script, use a text editor, such as
\tool{vim} or \tool{emacs}, to add the source request to your existing
shell-startup environment (i.e., to the \file{.tcshrc} file in your home directory). 

csh/\tool{tcsh}:
Sample \file{.tcshrc} file:
\begin{verbatim}
# Speed environment set up 
if ($HOSTNAME == speed-submit.encs.concordia.ca) then
   source /local/pkg/uge-8.6.3/root/default/common/settings.csh
endif
\end{verbatim}

Bourne shell/\tool{bash}:
Sample \file{.bashrc} file:
\begin{verbatim}
# Speed environment set up 
if [ $HOSTNAME = "speed-submit.encs.concordia.ca" ]; then
    . /local/pkg/uge-8.6.3/root/default/common/settings.sh
    printenv ORGANIZATION | grep -qw ENCS || . /encs/Share/bash/profile
fi
\end{verbatim}

Note that you will need to either log out and back in, or execute a new shell, 
for the environment changes in the updated \file{.tcshrc} or \file{.bashrc} file to be applied 
(\textbf{important}).
