\documentclass{easychair}

% https://en.wikibooks.org/wiki/LaTeX/Source_Code_Listings
\usepackage{listings}

% For inline citations
\usepackage{bibentry}
\nobibliography*

\makeatletter
\providecommand*{\input@path}{}
\g@addto@macro\input@path{{../src/}{src/}}% append
\makeatother

% My own commands, commands adapted from Joey and Peter

% Cross-reference commands.
% Per Dr. Grogono and my own self.
\newcommand{\xf}[1]{Figure~\ref{#1}}
\newcommand{\xp}[1]{page~\pageref{#1}}
\newcommand{\xs}[1]{Section~\ref{#1}}
\newcommand{\xa}[1]{Appendix~\ref{#1}}
\newcommand{\xc}[1]{Chapter~\ref{#1}}
\newcommand{\xt}[1]{Table~\ref{#1}}
\newcommand{\xl}[1]{Listing~\ref{#1}}

%
% Abbrs
%

\newcommand{\rpc}{{RPC\index{RPC}}}
\newcommand{\rmi}{{RMI\index{RMI}}}
\newcommand{\clp}{{CLP\index{CLP}}}
\newcommand{\tlp}{{TLP\index{TLP}}}
\newcommand{\slp}{{SLP\index{SLP}}}
\newcommand{\complus}{{DCOM+\index{DCOM+}}}
\newcommand{\corba}{{CORBA\index{CORBA}}}
\newcommand{\jini}{{Jini\index{Jini}}}
\newcommand{\jms}{{JMS\index{JMS}}}
\newcommand{\dotnet}{{.NET Remoting\index{.NET Remoting}}}
\newcommand{\gnu}{{GNU\index{GNU}}}
\newcommand{\tcpip}{{TCP/IP\index{TCP/IP}}}
\newcommand{\AST}{{AST\index{AST}}}


%
% The GIPSY
%

\newcommand{\gipc}{{GIPC\index{GIPC}\index{Frameworks!GIPC}}}
\newcommand{\gicf}{{GICF\index{GICF}\index{Frameworks!GICF}}}
\newcommand{\iplcf}{{IPLCF\index{IPLCF}\index{Frameworks!IPLCF}}}
\newcommand{\gee}{{GEE\index{GEE}\index{Frameworks!GEE}}}
\newcommand{\geer}{{GEER\index{GEER}}}
\newcommand{\gipsy}{{GIPSY\index{GIPSY}}}
\newcommand{\agipsy}{{AGIPSY\index{AGIPSY}}}
\newcommand{\ripe}{{RIPE\index{RIPE}\index{Frameworks!RIPE}}}
\newcommand{\dpr}{{DPR\index{DPR}}}
\newcommand{\dms}{{DMS\index{DMS}}}
\newcommand{\dmf}{{DMF\index{DMF}\index{Frameworks!DMF}}}
\newcommand{\dfg}{{DFG\index{DFG}}}


%
% The Lucids Family
%

\newcommand{\glu}{{GLU\index{GLU}}}
\newcommand{\glusharp}{{GLU\#\index{GLU\#}}}
\newcommand{\gipl}{{GIPL\index{GIPL}}}
\newcommand{\sipl}{{SIPL\index{SIPL}}}
\newcommand{\ipl}{{IPL\index{IPL}}}
\newcommand{\lucid}{{Lucid\index{Lucid}}}
\newcommand{\ilucid}{{Indexical Lucid\index{Indexical Lucid}}}
\newcommand{\jlucid}{{JLucid\index{JLucid}}}
\newcommand{\olucid}{{Objective Lucid\index{Tensor Lucid}}}
\newcommand{\tlucid}{{Tensor Lucid\index{Tensor Lucid}}}
\newcommand{\plucid}{{Partial Lucid\index{Partial Lucid}}}
\newcommand{\flucid}{{Forensic Lucid\index{Forensic Lucid}}}
\newcommand{\onyx}{{Onyx\index{Onyx}}}
\newcommand{\lucx}{{Lucx\index{Lucx}}}
\newcommand{\ooip}{{OOIP\index{OOIP}}}
\newcommand{\ioop}{{IOOP\index{IOOP}}}
\newcommand{\jooip}{{JOOIP\index{JOOIP}}}

%
% The Other Intensionals
%

\newcommand{\marfl}{{MARFL\index{MARFL}}}


%
% The Imperatives
%

\newcommand{\C}{{C\index{C}}}
\newcommand{\cpp}{{C++\index{C++}}}
\newcommand{\perl}{{Perl\index{Perl}}}
\newcommand{\java}{{Java\index{Java}}}
\newcommand{\python}{{Python\index{Python}}}
\newcommand{\fortran}{{Fortran\index{Fortran}}}
\newcommand{\aspectj}{{AspectJ\index{AspectJ}}}
\newcommand{\php}{{PHP\index{PHP}}}


%
% The Functionals
%

\newcommand{\lisp}{{LISP\index{LISP}}}
\newcommand{\scheme}{{Scheme\index{Scheme}}}
\newcommand{\haskell}{{Haskell\index{Haskell}}}
\newcommand{\mllessequal}{{ML$_{\le}$\index{ML$_{\le}$}}}
\newcommand{\fcpp}{{FC++\index{FC++}}}


%
% Lucid Operators: The Original and The New
%

\newcommand{\olucidop}[1]{{\bf \texttt{\textmd{\textsc{#1}}}}}
\newcommand{\lucidop}[1]{{\bf \texttt{#1}}}


%
% Forensic terms
%

% Forward transition
\newcommand{\trans}{$\psi$}
\newcommand{\transeq}[2]{$\psi(#1) = #2$}
% Inverse transition function
\newcommand{\invtrans}{$\Psi^{-1}$}
\newcommand{\invtranseq}[2]{$\Psi^{-1}(#1) = #2$}


%
% Util
%

\newcommand{\tab}[1]{\hspace{#1pt}}

\newcommand{\shrule}[0]{\vspace{3pt}\hrule\vspace{6pt}}
\newcommand{\ehrule}[0]{\vspace{6pt}\hrule\vspace{3pt}}

\newcommand{\nonterminal}[1]{$\mathtt{<\!\!#1\!\!>}$}

\newcommand{\source}[1]
{
	{\shrule}
	\scriptsize
	#1
	\normalsize
	\hrule
}

\newcommand{\sourcefloat}[3]
{
	\begin{figure}[!hp]
	\begin{centering}
	\begin{minipage}{0.5\textwidth}
	\source{#1}
	\end{minipage}
	\caption{\small{#3}}
	\label{#2}
	\end{centering}
	\end{figure}
}

\newcommand{\todo}[0]
{
	\begin{center}{\Large [TODO]}\index{TODO}\end{center}
}

\newcommand{\file}[1]{\url{#1}\index{Files!#1}}
\newcommand{\tool}[1]{\texttt{#1}\index{Tools!#1}}
\newcommand{\option}[1]{\texttt{#1}\index{Options!#1}}
\newcommand{\api}[1]{\texttt{#1}\index{API!#1}}
\newcommand{\apipackage}[1]{\url{#1}\index{API!Packages!#1}\index{Packages!#1}}
\newcommand{\datatype}[1]{\texttt{#1}\index{Type!#1}}
\newcommand{\codesegment}[1]{\texttt{\##1}\index{Segments!\##1}}


%
% Tools
%

\newcommand{\javacc}[0]{JavaCC\index{Tools!JavaCC}}
\newcommand{\junit}[0]{JUnit\index{Tools!JUnit}}


%
% Frameworks, APIs, Libraries
%

\newcommand{\marf}[0]{MARF\index{MARF}\index{Frameworks!MARF}\index{Libraries!MARF}}
\newcommand{\dmarf}[0]{DMARF\index{MARF!Distributed}\index{Frameworks!Distributed MARF}\index{Libraries!Distributed MARF}}
\newcommand{\admarf}
	[0]
	{ADMARF%
	\index{ADMARF}%
	\index{MARF!Autonomic}%
	\index{DMARF!Autonomic}%
	\index{Frameworks!Autonomic Distributed MARF}%
	\index{Libraries!Autonomic Distributed MARF}%
	}
\newcommand{\jdsf}[0]{JDSF\index{Frameworks!JDSF}\index{Libraries!JDSF}}
\newcommand{\sqlrand}[0]{SQLrand\index{SQLrand}}
\newcommand{\hsqldb}[0]{HSQLDB\index{HSQLDB}\index{Tools!HSQLDB}\index{Databases!HSQLDB}}
\newcommand{\cryptolysis}[0]{Cryptolysis\index{Frameworks!Cryptolysis}}
\newcommand{\assl}{ASSL\index{ASSL}\index{Autonomic Systems Specification Language}}


%
% Def
%

\newcommand{\statement}[2]
{
	\vspace{7pt}
	\shrule
	{\bf #1}

	#2
	\ehrule
	\vspace{7pt}
}

% \newcommand{\proposition}[2]
\newcommand{\sproposition}[2]
{
	\statement{Proposition #1}{#2}
}

% \newcommand{\definition}[2]
\newcommand{\sdefinition}[2]
{
	\statement{Definition #1}{#2}
}

% \newcommand{\axiom}[2]
\newcommand{\saxiom}[2]
{
	\statement{Axiom #1}{#2}
}

% \newcommand{\theorem}[2]
\newcommand{\stheorem}[2]
{
	\statement{Theorem #1}{#2}
}

%
% OS
%

\newcommand{\unix}{\index{Unix@{\sc{Unix}}}{\sc{Unix}}}
\newcommand{\macos}[1]{\index{Mac OS #1@{\sc{Mac OS #1}}}{\sc{Mac OS #1}}}
\newcommand{\linux}{\index{Linux@{\sc{Linux}}}{\sc{Linux}}}
\newcommand{\rhl}[1]{\index{Red Hat Linux #1@{\sc{Red Hat Linux #1}}}{\sc{Red Hat Linux #1}}}
\newcommand{\fcore}[1]{\index{Fedora Core #1@{\sc{Fedora Core #1}}}{\sc{Fedora Core #1}}}
\newcommand{\ubuntu}[1]{\index{Ubuntu #1@{\sc{Ubuntu #1}}}{\sc{Ubuntu #1}}}
\newcommand{\debian}[1]{\index{Debian #1@{\sc{Debian #1}}}{\sc{Debian #1}}}
\newcommand{\solaris}[1]{\index{Solaris #1@{\sc{Solaris #1}}}{\sc{Solaris #1}}}
\newcommand{\win}[1]{\index{Windows #1@{\sc{Windows #1}}}{\sc{Windows #1}}}


% Joey:

\newtheorem{defn}{Definition}
\newtheorem{axioms}{Axiom}
% \newtheorem{lemma}{Lemma}
\newtheorem{lemmas}{Lemma}
% \newcommand{\web}{{WWW}}
\newcommand{\wwweb}{{WWW}}
\newcommand{\bic}{{\index{BIC}BIC}}
\newcommand{\mni}{{\index{MNI}MNI}}
\newcommand{\nfs}{{\index{NFS}NFS}}
\newcommand{\crim}{{\index{CRIM}CRIM}}
\newcommand{\animal}{\index{Animal@{\sc{Animal}}}{\sc{Animal}}}
\newcommand{\paranimal}{\index{Paranimal@{\sc{ParAnimal}}}{\sc{ParAnimal}}}
\newcommand{\minc}{{\sc{MINC}}}
\newcommand{\netcdf}{{\sc{NetCDF}}}
\newcommand{\sgi}{{\index{SGI}}SGI}
\newcommand{\vv}{{\tt{*var}}}
\newcommand{\vd}{{\tt{?var}}}
\newcommand{\tv}{{\tt{*term}}}
\newcommand{\td}{{\tt{?term}}}
\newcommand{\fv}{{\tt{*fn}}}
\newcommand{\fd}{{\tt{?fn}}}
\newcommand{\home}{{\tt{home}}}
\newcommand{\light}{{\tt{light}}}
\newcommand{\heavy}{{\tt{heavy}}}
\newcommand{\lucidA}[1]{${\mathit{Lucid}}(#1)$}
\newcommand{\lucidL}[1]{{$\mathit{Lucid}$}($L$) }
\newcommand{\tristan}{\index{Tristan}Tristan}
\newcommand{\commercial}[1]{#1}
\newcommand{\al}{\mbox{$\alpha$}}
\newcommand{\be}{\mbox{$\beta$}}
\newcommand{\ga}{\mbox{$\gamma$}}
\newcommand{\vx}[1]{\mbox{$\overrightarrow{#1}$}}
\newcommand{\lvx}[1]{\mbox{$\mid\!\!\overrightarrow{#1}\!\!\mid$}}
\newcommand{\svx}[1]{{\small \mbox{$\overrightarrow{#1}$}}}
\newcommand{\curl}[1]{\nabla\times\;\mathbf{#1}}
\newcommand{\components}[3]{{_{#3}}{#2}_{#1}}
\newcommand{\componentsp}[3]{{_{#3}}{#2}'_{#1}}
\newcommand{\mypageheader}[1]{\vspace*{22mm}{\Huge \bf #1}\vspace*{5mm}}
\newcommand{\myfig}[1]{\center{\makebox[\textwidth]{\hbox{\vbox{\epsfbox{#1}}}}}}
%\newcommand{\myfig}[1]{\makebox[\textwidth]{\hbox{\vbox{60mm}}}}
\newcommand{\ctxt}{{\mathcal L},{\mathcal D},{\mathcal P},{\mathcal W}}
\newcommand{\noWctxt}{{\mathcal L},{\mathcal D},{\mathcal P}}
\newcommand{\myvdash}{\:\vdash\:}
\newcommand{\mysemi}{\::\:}
\newcommand{\Spc}          {{\mathcal{S}}}
\newcommand{\corner}[1]    {\ulcorner #1\urcorner}
\newcommand{\db}[1]        {\{#1\}}
\newcommand{\mtt}[1]       {{\mathtt{#1}}}
\newcommand{\mrm}[1]       {{\mathrm{#1}}}
\newcommand{\mem}[1]       {{\mathit{#1}}}

\newcommand{\mathfbyd}     {{\mathtt{fby.d}}}
\newcommand{\mathfirstd}   {{\mathtt{first.d}}}
\newcommand{\mathnextd}    {{\mathtt{next.d}}}
\newcommand{\mathprevd}    {{\mathtt{prev.d}}}
\newcommand{\mathwvrd}     {{\mathtt{wvr.d}}}
\newcommand{\mathasad}     {{\mathtt{asa.d}}}
\newcommand{\mathupond}    {{\mathtt{upon.d}}}
\newcommand{\mathfby}      {{\mathtt{fby}}}
\newcommand{\mathbefore}   {{\mathtt{before}}}
\newcommand{\mathfirst}    {{\mathtt{first}}}
\newcommand{\mathnext}     {{\mathtt{next}}}
\newcommand{\mathprev}     {{\mathtt{prev}}}
\newcommand{\mathwvr}      {{\mathtt{wvr}}}
\newcommand{\mathasa}      {{\mathtt{asa}}}
\newcommand{\mathupon}     {{\mathtt{upon}}}
\newcommand{\mathif}       {{\mathtt{if}}}
\newcommand{\maththen}     {{\mathtt{then}}}
\newcommand{\mathelse}     {{\mathtt{else}}}
\newcommand{\mathfi}       {{\mathtt{fi}}}
\newcommand{\mathatd}      {{\mathtt{@.d}}}
\newcommand{\mathat}       {{\mathtt{@.}}}
\newcommand{\mathtagd}     {{\mathtt{\#.d}}}
\newcommand{\mathtag}      {{\mathtt{\#.}}}
\newcommand{\mathwhere}    {{\mathtt{where}}}
\newcommand{\mathdimension}{{\mathtt{dimension}}}
\newcommand{\mathhome}	   {{\mathtt{home}}}
\newcommand{\mathheavy}	   {{\mathtt{heavy}}}
\newcommand{\mathlight}	   {{\mathtt{light}}}
\newcommand{\mathiseod}    {{\mathtt{iseod}}}
\newcommand{\mathiserror}  {{\mathtt{iserror}}}
\newcommand{\mathend}      {{\mathtt{end}}}
\newcommand{\matheod}      {{\mathtt{eod}}}
\newcommand{\matherror}    {{\mathtt{error}}}
\newcommand{\mathtrue}     {{\mathtt{true}}}
\newcommand{\mathfalse}    {{\mathtt{false}}}
\newcommand{\Ek}           {${\mathbf{E_{k}}}$}
\newcommand{\Eop}           {${\mathbf{E_{op}}}$}
\newcommand{\Eid}           {${\mathbf{E_{id}}}$}
\newcommand{\Efid}          {${\mathbf{E_{fid}}}$}
\newcommand{\Econdt}        {${\mathbf{E_{c_{T}}}}$}
\newcommand{\Econdf}        {${\mathbf{E_{c_{F}}}}$}
\newcommand{\Ewhere}        {${\mathbf{E_{w}}}$}
\newcommand{\Eat}           {${\mathbf{E_{at}}}$}
\newcommand{\Etag}          {${\mathbf{E_{tag}}}$}
\newcommand{\Qid}           {${\mathbf{Q_{id}}}$}
\newcommand{\Qfid}          {${\mathbf{Q_{fid}}}$}
\newcommand{\QQ}            {${\mathbf{QQ}}$}
\newcommand{\const}        {{\mathit{k}}}
\newcommand{\varid}        {{\mathit{id}}}
\newcommand{\dimid}        {{\mathit{did}}}
\newcommand{\letter}       {{\mathit{letter}}}
\newcommand{\digit}        {{\mathit{digit}}}
\newcommand{\character}    {{\mathit{char}}}
\newcommand{\mystring}     {{\mathit{string}}}
\newcommand{\boolean}      {{\mathit{boolean}}}
\newcommand{\real}         {{\mathit{real}}}
\newcommand{\ASCIIchar}    {{\mathit{ASCIIchar}}}
\newcommand{\alphanum}     {{\mathit{alphanum}}}
\newcommand{\integer}      {{\mathit{integer}}}
\newcommand{\E}            {{\mathit{E}}}
%\renewcommand{\E}            {{\mathit{E}}}
\newcommand{\userfct}      {{\mathit{userfct}}}
\newcommand{\llop}         {{\textit{intensional-op}}}
\newcommand{\luop}         {{\textit{i-unary-op}}}
\newcommand{\lbop}         {{\textit{i-binary-op}}}
\newcommand{\op}           {{\textit{data-op}}}
\newcommand{\uop}          {{\textit{unary-op}}}
\newcommand{\bop}          {{\textit{binary-op}}}
\newcommand{\ifexpr}       {{\mathit{ifexpr}}}
\newcommand{\deflist}      {{\mathit{deflist}}}
%\newcommand{\dimdef}       {{\mathit{dimdef}}}
\newcommand{\fctid  }      {{\mathit{fid}}}
\newcommand{\tensorid}[2]  {{\mathit{tid_{#1}#2}}}
\newcommand{\usc}          {\mathit{\raisebox{0mm}{\_}}}
\newcommand{\dimlist}      {{\mathit{dimlist}}}
\newcommand{\Elist}        {{\mathit{Elist}}}
\newcommand{\simpleuop}    {{\mathit{mathuop}}}
\newcommand{\complexuop}   {{\mathit{intuop}}}
\newcommand{\defmy}        {{\mathit{Q}}}
\newcommand{\paramlist}    {{\mathit{parlist}}}
\newcommand{\id}           {{\mathit{identifier}}}
\newcommand{\Luciduop}     {{\mathit{Luciduop}}}
\newcommand{\Lucidbop}     {{\mathit{Lucidbop}}}
\newcommand{\simplebop}    {{\mathit{mathbop}}}
\newcommand{\complexbop}   {{\mathit{intbop}}}
\newcommand{\arithbop}     {{\textit{arith-op}}}
\newcommand{\relbop}       {{\textit{rel-op}}}
\newcommand{\logbop}       {{\textit{log-op}}}
\newcommand{\bitbop}       {{\textit{bit-op}}}
\newcommand{\seqbop}       {{\textit{seq-op}}}
\newcommand{\B}{\!\!\!\!\!\!\!\!\!\!\!\!\!\!\!\!}
\newcommand{\Bs}{\!\!\!}
\newcommand{\Bt}{\!}
\newcommand{\Dim}{{\mathcal{D}}}
\newcommand{\Point}{{\mathcal{P}}}
\newcommand{\PointP}{{\mathcal{P}}\!\dagger\!}
\newcommand{\Tag}{{\mathcal{T}}}
\newcommand{\Lang}{{\mathcal{L}}}
\newcommand{\Def}{{\mathcal{D}}}
\newcommand{\Ware}{{\mathcal{W}}}
\newcommand{\WareD}[2]{{\mathcal{W}}?\!\left\{[#1]#2\right\}}
\newcommand{\WareP}[3]{{\mathcal{W}}\!\dagger\!\left\{[#1]#2:#3\right\}}
\newcommand{\Id}{{\mathcal{I}}}
\newcommand{\Val}{{\mathcal{V}}}
\newcommand{\Stream}{{\mathcal{I}}}
\newcommand{\Expr}{{\mathcal{E}}}
\newcommand{\allExpr}{\Expr^\infty}
\newcommand{\allDim}{\Delta^\infty}
\newcommand{\allPoint}{\Pi^\infty}
\newcommand{\allStream}{\Stream^\infty}
\newcommand{\allVal}{\Val^\infty}
\newcommand{\allTag}{\Tag^\infty}
\newcommand{\extdef}{\stackrel{ext}{\equiv}}
\newcommand{\Sb}{\mathbf{Sb}}
\newcommand{\Sw}{\mathbf{Sw}}

%\newenvironment{program}
%		{\begin{quote}}
%		{\end{quote}}
\newtheorem{mydef}
		{{\bf Definition:}}
		{}
\newcommand{\paracite}[2]
		{\vspace{0.5cm}
		{\it{#1

		}}
		{\begin{flushright}---#2\end{flushright}}
}
\newcommand{\cutecite}[2]
		{\vspace{0.5cm}
		{\begin{flushright}
		{\it{#1}}\\
		---#2
		\end{flushright}}
}
\newcommand{\sembox}[3]
		{\TR{   \begin{small}
			\begin{tabular}{|p{4mm}|c|}\hline
			$\!\!$#1 & {\tt{#2}}\\\cline{2-2}
		   	   & [#3]\\\hline
			\end{tabular}
			\end{small}

		}}
%\floatstyle{boxed}
%\restylefloat{table}
%\restylefloat{figure}
%\floatname{boxtable}{Table}
%\newfloat{boxtable}{h}{lot}[chapter]


%\newcounter{definition}
%\setcounter{definition}{0}
%\newenvironment{definition}
%{
%\parindent0mm
%\parskip3mm
%\addtocounter{definition}{1}
%{\bf Definition \arabic{definition}}:
%}

%\newcounter{theorem}
%\setcounter{theorem}{0}
%\newenvironment{theorem}
%{
%\parindent0mm
%\parskip3mm
%\addtocounter{theorem}{1}
%{\bf Theorem \arabic{theorem}}:
%}

%\newtheorem{proposition}{Proposition}

\def\mymid{\vrule depth 4pt height 10pt width 0.2mm}
\def\myspace{\hspace*{3mm}}
\def\mymidspace{\mymid\myspace}
\def\myvert{\raise 2.27pt \hbox{\vrule depth 0pt height 8pt width 0.2mm}}
\def\myarrow{\hspace*{0.43mm}%
             \raise 2.29pt\hbox{\vrule depth 0pt height 8pt width 0.16mm}%
             \hspace*{-0.32mm}%
             $\longrightarrow$
             \ %
             }
\def\mmyarrow{$\rightarrow$\ }

%\psset{unit=.75cm}

\newcommand{\johndef}{\mathcal{D}}
\newcommand{\johnjvmdef}{\mathcal{D}_{jvm}}
\newcommand{\johntdef}{\mathcal{T}}
\newcommand{\myid}{\textit{id}}
\newcommand{\mytid}{\textit{tid}}
\newcommand{\mydagger}{\!\dagger\!}
\newcommand{\context}[2]{\mathcal{D},\mathcal{P} \vdash #1 : #2}
\newcommand{\jvmcontext}[2]{\mathcal{D}_{jvm} \vdash #1 : #2}
\newcommand{\pcontext}[2]{\mathcal{D},\mathcal{P},\mathcal{N} \vdash #1 : #2}
\newcommand{\contextW}[2]{\mathcal{D},\mathcal{P},\mathcal{W} \vdash #1 : #2}
\newcommand{\contextWp}[2]{\mathcal{D},\mathcal{P},\mathcal{W}' \vdash #1 : #2}
\newcommand{\qcontext}[2]{\mathcal{D},\mathcal{P} \vdash #1 \::\: #2}
\newcommand{\qjvmcontext}[2]{\mathcal{D}_{jvm} \vdash #1 \::\: #2}
\newcommand{\pqcontext}[2]{\mathcal{D},\mathcal{P},\mathcal{N} \vdash #1 \::\: #2}
\newcommand{\qcontextW}[2]{\mathcal{D},\mathcal{P,\mathcal{W}} \vdash #1 \::\: #2}
\newcommand{\myifthenelse}{\mathtt{if}\;E\;\mathtt{then}\;E'\;\mathtt{else}\;E''}

\def\Lfirst{\index{first@{\texttt{first}}}\texttt{first}\;}
\def\Lnext{\index{next@{\texttt{next}}}\texttt{next}\;}
\def\Lfby{\index{fby@{\texttt{fby}}}\;\texttt{fby}\;}
\def\Lat{\index{a@{\texttt{\char64}}}\;\texttt{\char64}\;}
\def\LSat{\index{a@{\texttt{\char64}}}\texttt{\char64}}
\def\Lhash{\index{a@{\texttt{\char35}}}\texttt{\char35}}
\def\Lwvr{\index{wvr@{\texttt{wvr}}}\;\texttt{wvr}\;}
\def\Lupon{\index{upon@{\texttt{upon}}}\;\texttt{upon}\;}
\def\LSupon{\index{upon@{\texttt{upon}}}\;\texttt{upon}}
\def\Lasa{\index{asa@{\texttt{asa}}}\;\texttt{asa}\;}
\def\Leod{\index{eod@{\texttt{eod}}}\texttt{eod}}
\def\Liseod{\index{iseod@{\texttt{iseod}}}\texttt{iseod}}
\def\Lif{\index{ifthenelse@{\texttt{if then else}}}\texttt{if}\;}
\def\Lthen{\;\texttt{then}\;}
\def\Lelse{\;\texttt{else}\;}
\def\Lsif{\index{ifthenelse@{\texttt{if then else}}}\texttt{\scriptsize if}\;}
\def\Lsthen{\;\texttt{\scriptsize then}\;}
\def\Lselse{\;\texttt{\scriptsize else}\;}

\def\mufirst{\index{first@{\texttt{first}}}\mathrm{\underline{\mathtt{first}}}\;}
\def\munext{\index{next@{\texttt{next}}}\mathrm{\underline{\mathtt{next}}}\;}
\def\mufby{\index{fby@{\texttt{fby}}}\;\mathrm{\underline{\mathtt{fby}}}\;}
\def\muwvr{\index{wvr@{\texttt{wvr}}}\;\mathrm{\underline{\mathtt{wvr}}}\;}
\def\muupon{\index{upon@{\texttt{upon}}}\;\mathrm{\underline{\mathtt{upon}}}\;}
\def\muasa{\index{asa@{\texttt{asa}}}\;\mathrm{\underline{\mathtt{asa}}}\;}

\def\mfirst{\index{first@{\texttt{first}}}\mathrm{{\mathtt{first}}}\;}
\def\mprev{\index{prev@{\texttt{prev}}}\mathrm{{\mathtt{prev}}}\;}
\def\mnext{\index{next@{\texttt{next}}}\mathrm{{\mathtt{next}}}\;}
\def\mfby{\index{fby@{\texttt{fby}}}\;\mathrm{{\mathtt{fby}}}\;}
\def\mwvr{\index{wvr@{\texttt{wvr}}}\;\mathrm{{\mathtt{wvr}}}\;}
\def\mupon{\index{upon@{\texttt{upon}}}\;\mathrm{{\mathtt{upon}}}\;}
\def\masa{\index{asa@{\texttt{asa}}}\;\mathrm{{\mathtt{asa}}}\;}

\def\Tfirst{\index{first@{\texttt{first}}}\texttt{first}}
\def\Tnext{\index{next@{\texttt{next}}}\texttt{next}}
\def\Tfby{\index{fby@{\texttt{fby}}}\texttt{fby}}
\def\Twvr{\index{wvr@{\texttt{wvr}}}\texttt{wvr}}
\def\Tupon{\index{upon@{\texttt{upon}}}\texttt{upon}}
\def\Tasa{\index{asa@{\texttt{asa}}}\texttt{asa}}

\newcommand{\eqdef}{\stackrel{{\mathrm{def}}}{=}}
\newcommand{\mylinebefore}{\noindent\rule{.1mm}{3mm}\rule[3mm]{.995\textwidth}{.1mm}\rule{.1mm}{3mm}\vspace*{-5mm}}
\newcommand{\mylineafter}{\vspace*{-5mm}\noindent\rule{.1mm}{3mm}\rule{.995\textwidth}{.1mm}\rule{.1mm}{3mm}}
\newcommand{\myprop}[1]{
\mylinebefore
\begin{proposition}
#1
\end{proposition}
\mylineafter
}


%% Document
%%
\begin{document}

% ------------------------------------------------------------------------------
%% Front Matter
%%
% Regular title as in the article class.
%
\title{Speed: The GCS ENCS Cluster}

% \titlerunning{} has to be set to either the main title or its shorter
% version for the running heads. Use {\sf} for highlighting your system
% name, application, or a tool.
%
\titlerunning{Speed: The GCS ENCS Cluster}

% Previously VI
%\date{Version 6.5}
\date{\textbf{Version 6.6-dev-07}}

% Authors are joined by \and and their affiliations are on the
% subsequent lines separated by \\ just like the article class
% allows.
%
\author{
    Serguei A. Mokhov
\and
    Gillian A. Roper
\and
    Network, Security and HPC Group\footnote{The group acknowledges the initial manual version VI produced by Dr.~Scott Bunnell while with us.}\\
    \affiliation{Gina Cody School of Engineering and Computer Science}\\
    \affiliation{Concordia University}\\
    \affiliation{Montreal, Quebec, Canada}\\
    \affiliation{\url{rt-ex-hpc~AT~encs.concordia.ca}}\\
}

% \authorrunning{} has to be set for the shorter version of the authors' names;
% otherwise a warning will be rendered in the running heads.
%
\authorrunning{Mokhov, Roper, NAG/HPC, GCS ENCS}
\indexedauthor{Mokhov, Serguei}
\indexedauthor{Roper, Gillian}
\indexedauthor{NAG/HPC}

%%%%%%%%%%%%%%%%%%%%%%%%%%%%%%%%%%%%%%%%%%%%%%%%%%%
\maketitle
%%%%%%%%%%%%%%%%%%%%%%%%%%%%%%%%%%%%%%%%%%%%%%%%%%%

% ------------------------------------------------------------------------------
\begin{abstract}
This document primarily presents a quick start
guide to the usage of the Gina Cody School of
Engineering and Computer Science compute server farm
called ``Speed'' -- the GCS ENCS Speed cluster,
managed by HPC/NAG of GCS ENCS, Concordia University,
Montreal, Canada.
\end{abstract}

% ------------------------------------------------------------------------------
\tableofcontents
\clearpage

% ------------------------------------------------------------------------------
\section{Introduction}

This document contains basic information required to use ``Speed'' as well as 
tips and tricks, examples, and references to projects and papers that have used Speed. 
User contributions of sample jobs and/or references are welcome. 
Details are sent to the \texttt{hpc-ml} mailing list.

% ------------------------------------------------------------------------------
\subsection{Resources}

\begin{itemize}
\item
Our public GitHub page where the manual and sample job scripts
are maintained (pull-requests (PRs), subject to review, are welcome):\\
\url{https://github.com/NAG-DevOps/speed-hpc}\\
\url{https://github.com/NAG-DevOps/speed-hpc/pulls}

\item
PDF version of this manual:\\
\url{https://github.com/NAG-DevOps/speed-hpc/blob/master/doc/speed-manual.pdf}\\
HTML version of this manual:\\
\url{https://nag-devops.github.io/speed-hpc/}

\item
Our official Concordia page for the ``Speed'' cluster:\\
\url{https://www.concordia.ca/ginacody/aits/speed.html}\\
which includes access request instructions.

\item
All Speed users are subscribed to the \texttt{hpc-ml} mailing
list.

\item
\href
	{https://docs.google.com/presentation/d/1zu4OQBU7mbj0e34Wr3ILXLPWomkhBgqGZ8j8xYrLf44}
	{Speed Server Farm Presentation 2022}~\cite{speed-intro-preso}.

\end{itemize}

% ------------------------------------------------------------------------------
\subsection{Team}

\begin{itemize}
	\item 
Serguei Mokhov, Manager, Networks, Security and HPC
	\item 
Gillian Roper, Senior Administrator, System, Information Technology
	\item 
Carlos Alarcón Meza, Administrator, System, High Performance Computing and Networking, Information Technology
	\item 
Tariq Daradkeh, PhD, IT Instructional Specialist, Information Technology
\end{itemize}

We receive support from the rest of AITS teams, such
as NAG, SAG, FIS, and DOG.

% ------------------------------------------------------------------------------
\subsection{What Speed Comprises}

\begin{itemize}
\item
Twenty four (24) 32-core compute nodes, each with 512 GB of memory and approximately 1 TB of volatile-scratch disk space. 
\item
Twelve (12) NVIDIA Tesla P6 GPUs, with 16 GB of memory (compatible with the CUDA, OpenGL, OpenCL, and Vulkan APIs). 
\item
One AMD FirePro S7150 GPUs, with 8 GB of memory (compatible with the Direct X, OpenGL, OpenCL, and Vulkan APIs). 
\end{itemize}

% ------------------------------------------------------------------------------
\subsection{What Speed Is Ideal For}
\label{sect:speed-is-for}

\begin{itemize}
\item
To design and develop, test and run parallel, batch, and other algorithms, scripts with partial data sets.
\item
Prepare them for big clusters:
	\begin{itemize}
	\item 
	Calcul Quebec and Compute Canada
	\item 
	Cloud platforms
	\end{itemize}
\item
Jobs that are too demanding for a desktop. 
\item
Single-core batch jobs; multithreaded jobs up to 32 cores (i.e., a single machine).
\item
Anything that can fit into a 500-GB memory space and a scratch space of approximately 1 TB. 
\item
CPU-based jobs. 
\item
CUDA GPU jobs (\texttt{speed-05}, \texttt{speed-17}).
\item
Non-CUDA GPU jobs using OpenCL (\texttt{speed-19} and \texttt{speed-05|17}).
\end{itemize}

% ------------------------------------------------------------------------------
\subsection{What Speed Is Not}
\label{sect:speed-is-not}

\begin{itemize}
\item Speed is not a web host and does not host websites.
\item Speed is not meant for CI automation deployments for Ansible or similar tools. 
\item Does not run Kubernetes or other container orchestration software.
\item Does not run Docker. (Note: Speed does run Singularity and many Docker containers can be converted to Singularity containers with a single command.)
\item Speed is not for jobs executed outside of the scheduler. (Jobs running outside of the scheduler will be killed and all data lost.)
\end{itemize}

% ------------------------------------------------------------------------------
\subsection{Available Software}

We have a great number of open-source software available and installed
on Speed~--~various Python, CUDA versions, {\cpp}/{\java} compilers, OpenGL,
OpenFOAM, OpenCV, TensorFlow, OpenMPI, OpenISS, {\marf}~\cite{marf}, etc.
There are also a number of commercial packages, subject to licensing
contributions, available, such as MATLAB~\cite{matlab,scholarpedia-matlab}, Abaqus~\cite{abaqus}, 
Ansys, Fluent~\cite{fluent}, etc. 

To see the packages available, run \texttt{ls -al /encs/pkg/} on \texttt{speed.encs}.

In particular, there are over 2200 programs available in
\texttt{/encs/bin} and \texttt{/encs/pkg} under Scientific Linux 7 (EL7).

\begin{itemize}
	\item 
Popular concrete examples:
\begin{itemize}
	\item 
MATLAB (R2016b, R2018a, R2018b)
	\item 
Fluent (19.2)
	\item 
Singularity (Docker-like container), can run other OS's apps, like Ubuntu's, converted Docker containers.
\end{itemize}
	\item 
We do our best to accommodate custom software requests.
Python environments can be used to have user-custom installs
in the scratch directory.
	\item 
A number of specific environments are available, too.
	\item 
Popular examples mentioned (loaded with, \tool{module}):
\begin{itemize}
	\item 
Python (2.3.0 - 3.5.1)
	\item 
Gurobi (7.0.1, 7.5.0, 8.0.0, 8.1.0)
	\item 
Ansys (16, 17, 18, 19)
	\item 
OpenFOAM (2.3.1, 3.0.1, 5.0, 6.0)
	\item 
Cplex 12.6.x to 12.8.x
	\item 
OpenMPI 1.6.x, 1.8.x, 3.1.3
\end{itemize}
\end{itemize}

% ------------------------------------------------------------------------------
\subsection{Requesting Access}

After reviewing the ``What Speed is'' (\xs{sect:speed-is-for}) and
``What Speed is Not'' (\xs{sect:speed-is-not}), request access to the ``Speed'' 
cluster by emailing: \texttt{rt-ex-hpc AT encs.concordia.ca}.
%
Faculty and staff may request the access directly.
Students must include the following in their message:

\begin{itemize} 
	\item GCS ENCS username
	\item Name and email (CC) of the supervisor or instructor
	\item Written request from the supervisor or instructor for the ENCS username to be granted access to ``Speed''
\end{itemize}

% ------------------------------------------------------------------------------
\section{Job Management}
\label{sect:job-management}

In these instructions, anything bracketed like so, \verb+<>+, indicates a
label/value to be replaced (the entire bracketed term needs replacement).

% ------------------------------------------------------------------------------
\subsection{Getting Started}

Before getting started, please review the ``What Speed is'' (\xs{sect:speed-is-for})
and ``What Speed is Not'' (\xs{sect:speed-is-not}).
Once your GCS ENCS account has been granted access to ``Speed'',
use your GCS ENCS account credentials to create an SSH connection to 
\texttt{speed} (an alias for \texttt{speed-submit.encs.concordia.ca}). 

% ------------------------------------------------------------------------------
\subsubsection{SSH Connections}

Requirements to create connections to Speed:
\begin{enumerate}
	\item
An active \textbf{ENCS user account} which has permission to connect to Speed.
	\item
If you are off campus, an active connection to Concordia's VPN. Accessing Concordia's VPN requires a Concordia \textbf{netname}. 
	\item
Windows systems require a terminal emulator such as PuTTY (or MobaXterm). 
\end{enumerate}

Open up a terminal window and type in the following SSH command being sure to replace
\verb!<ENCSusername>! with your ENCS account's username.

\begin{verbatim}
ssh <ENCSusername>@speed.encs.concordia.ca
\end{verbatim}

All users are expected to have a basic understanding of Linux and its 
commonly used commands. 

% ------------------------------------------------------------------------------
\subsubsection{Environment Set Up}
\label{sect:envsetup}

After creating an SSH connection to ``Speed'', you will need to source 
the ``Altair Grid Engine (AGE)'' scheduler's settings file. 
Sourcing the settings file will set the environment variables required to 
execute scheduler commands.

Based on the UNIX shell type, choose one of the following commands to source
the settings file. 

csh/\tool{tcsh}:
\begin{verbatim}
source /local/pkg/uge-8.6.3/root/default/common/settings.csh 
\end{verbatim}

Bourne shell/\tool{bash}:
\begin{verbatim}
. /local/pkg/uge-8.6.3/root/default/common/settings.sh 
\end{verbatim}

In order to set up the default ENCS bash shell, executing the following command 
is also required:
\begin{verbatim}
printenv ORGANIZATION | grep -qw ENCS || . /encs/Share/bash/profile 
\end{verbatim}

To verify that you have access to the scheduler commands execute 
\texttt{qstat -f -u "*"}. If an error is returned, attempt sourcing 
the settings file again.

The next step is to copy a job template to your home directory and to set up your
cluster-specific storage. Execute the following command from within your
home directory. (To move to your home directory, type \texttt{cd} at the Linux
prompt and press \texttt{Enter}.) 

\begin{verbatim}
cp /home/n/nul-uge/template.sh . && mkdir /speed-scratch/$USER
\end{verbatim}

\textbf{Tip:} Add the source command to your shell-startup script. 

\textbf{Tip:} the default shell for GCS ENCS users is \tool{tcsh}.
If you would like to use \tool{bash}, please contact 
\texttt{rt-ex-hpc AT encs.concordia.ca}.

For \textbf{new ENCS Users}, and/or those who don't have a shell-startup script, 
based on your shell type use one of the following commands to copy a start up script 
from \texttt{nul-uge}'s. home directory to your home directory. (To move to your home
directory, type \tool{cd} at the Linux prompt and press \texttt{Enter}.)

csh/\tool{tcsh}:
\begin{verbatim}
cp /home/n/nul-uge/.tcshrc . 
\end{verbatim}

Bourne shell/\tool{bash}:
\begin{verbatim}
cp /home/n/nul-uge/.bashrc . 
\end{verbatim}

Users who already have a shell-startup script, use a text editor, such as
\tool{vim} or \tool{emacs}, to add the source request to your existing
shell-startup environment (i.e., to the \file{.tcshrc} file in your home directory). 

csh/\tool{tcsh}:
Sample \file{.tcshrc} file:
\begin{verbatim}
# Speed environment set up 
if ($HOSTNAME == speed-submit.encs.concordia.ca) then
   source /local/pkg/uge-8.6.3/root/default/common/settings.csh
endif
\end{verbatim}

Bourne shell/\tool{bash}:
Sample \file{.bashrc} file:
\begin{verbatim}
# Speed environment set up 
if [ $HOSTNAME = "speed-submit.encs.concordia.ca" ]; then
    . /local/pkg/uge-8.6.3/root/default/common/settings.sh
    printenv ORGANIZATION | grep -qw ENCS || . /encs/Share/bash/profile
fi
\end{verbatim}

Note that you will need to either log out and back in, or execute a new shell, 
for the environment changes in the updated \file{.tcshrc} or \file{.bashrc} file to be applied 
(\textbf{important}).

% ------------------------------------------------------------------------------
\subsection{Job Submission Basics}

Preparing your job for submission is fairly straightforward. Editing a copy
of the \file{template.sh} you moved into your home directory during
\xs{sect:envsetup} is a good place to start. You can also use a job script
example from our GitHub's (\url{https://github.com/NAG-DevOps/speed-hpc}) ``src'' 
directory and base your job on it.

Job scripts are broken into four main sections: 
\begin{itemize}
	\item Directives
	\item Module Loads
	\item User Scripting
\end{itemize}

% ------------------------------------------------------------------------------
\subsubsection{Directives}

Directives are comments included at the beginning of a job script that set the shell 
and the options for the job scheduler. 

The shebang directive is always the first line of a script. In your job script, 
this directive sets which shell your script's commands will run in. On ``Speed'', 
we recommend that your script use a shell from the \texttt{/encs/bin} directory. 

To use the \texttt{tcsh} shell, start your script with: \verb|#!/encs/bin/tcsh|

For \texttt{bash}, start with: \verb|#!/encs/bin/bash|

Directives that start with \verb|"#$"|, set the options for the cluster's 
``Altair Grid Engine (AGE)'' scheduler. The script template, \file{template.sh}, 
provides the essentials:

\begin{verbatim}
#$ -N <jobname>
#$ -cwd
#$ -m bea
#$ -pe smp <corecount>
#$ -l h_vmem=<memory>G
\end{verbatim}

Replace, \verb+<jobname>+, with the name that you want your cluster job to have;
\option{-cwd}, makes the current working directory the ``job working directory'',
and your standard output file will appear here; \option{-m bea}, provides e-mail
notifications (begin/end/abort); replace, \verb+<corecount>+, with the degree of
(multithreaded) parallelism (i.e., cores) you attach to your job (up to 32),
be sure to delete or comment out the \verb| #$ -pe smp | parameter if it 
is not relevant; replace, \verb+<memory>+, with the value (in GB), that you want 
your job's memory space to be (up to 500), and all jobs MUST have a memory-space 
assignment.

If you are unsure about memory footprints, err on assigning a generous
memory space to your job so that it does not get prematurely terminated
(the value given to \api{h\_vmem} is a hard memory ceiling). You can refine
\api{h\_vmem} values for future jobs by monitoring the size of a job's active
memory space on \texttt{speed-submit} with:

\begin{verbatim}
qstat -j <jobID> | grep maxvmem
\end{verbatim}

Memory-footprint values are also provided for completed jobs in the final
e-mail notification (as, ``Max vmem'').

\emph{Jobs that request a low-memory footprint are more likely to load on a busy
cluster.}

% ------------------------------------------------------------------------------
\subsubsection{Module Loads}

As your job will run on a compute or GPU ``Speed'' node, and not the submit node,
any software that is needed must be loaded by the job script. Software is loaded
within the script just as it would be from the command line.

To see a list of which modules are available, execute the following from the 
command line on \texttt{speed-submit}.

\begin{verbatim}
module avail
\end{verbatim}

To list for a particular program (\tool{matlab}, for example):

\begin{verbatim}
module -t avail matlab
\end{verbatim}

Which, of course, can be shortened to match all that start with a
particular letter:

\begin{verbatim}
module -t avail m
\end{verbatim}

Insert the following in your script to load the \tool{matlab/R2020a}) module:

\begin{verbatim}
module load matlab/R2020a/default
\end{verbatim}

Use, \option{unload}, in place of, \option{load}, to remove a module from active use.

To list loaded modules:

\begin{verbatim}
module list
\end{verbatim}

To purge all software in your working environment:

\begin{verbatim}
module purge
\end{verbatim}

Typically, only the \texttt{module load} command will be used in your script.

% ------------------------------------------------------------------------------
\subsubsection{User Scripting}

The last part the job script is the scripting that will be executed by the job. 
This part of the job script includes all commands required to set up and 
execute the task your script has been written to do. Any Linux command can be used 
at this step. This section can be a simple call to an executable or a complex 
loop which iterates through a series of commands.

Every software program has a unique execution framework. It is the responsibility 
of the script's author (e.g., you) to know what is required for the software used 
in your script by reviewing the software's documentation. Regardless of which software
your script calls, your script should be written so that the software knows the 
location of the input and output files as well as the degree of parallelism. 
Note that the cluster-specific environment variable, \api{NSLOTS}, resolves 
to the value provided to the scheduler in the \option{-pe smp} option. 

Jobs which touch data-input and data-output files more than once, should make use 
of \api{TMPDIR}, a scheduler-provided working space almost 1~TB in size.
\api{TMPDIR} is created when a job starts, and exists on the local disk of the
compute node executing your job. Using \api{TMPDIR} results in faster I/O operations 
than those to and from shared storage (which is provided over NFS). 

An sample job script using \api{TMPDIR} is available at \texttt{/home/n/nul-uge/templateTMPDIR.sh}: 
the job is instructed to change to \api{\$TMPDIR}, to make the new directory \texttt{input}, to copy data from
\texttt{\$SGE\_O\_WORKDIR/references/} to \texttt{input/} (\texttt{\$SGE\_O\_WORKDIR} represents the
current working directory), to make the new directory \texttt{results}, to
execute the program (which takes input from \texttt{\$TMPDIR/input/} and writes
output to \texttt{\$TMPDIR/results/}), and finally to copy the total end results
to an existing directory, \texttt{processed}, that is located in the current
working directory. TMPDIR only exists for the duration of the job, though,
so it is very important to copy relevant results from it at job's end.

% ------------------------------------------------------------------------------
\subsection{Sample Job Script}

Now, let's look at a basic job script, \file{tcsh.sh} in \xf{fig:tcsh.sh}
(you can copy it from our GitHub page or from \texttt{/home/n/nul-uge}).

\begin{figure}[htpb]
	\lstinputlisting[language=csh,frame=single,basicstyle=\ttfamily]{tcsh.sh}
	\caption{Source code for \file{tcsh.sh}}
	\label{fig:tcsh.sh}
\end{figure}

The first line is the shell declaration (also know as a shebang) and sets the shell to \emph{tcsh}.
The lines that begin with \texttt{\#\$} are directives for the scheduler.

\begin{itemize}
	\item \texttt{-N} sets \emph{qsub-test} as the jobname
	\item \texttt{-cwd} tells the scheduler to execute the job from the current working directory
	\item \texttt{-l h\_vmem=1GB} requests and assigns 1GB of memory to the job. CPU jobs \emph{require} the \texttt{-l h\_vmem} option to be set.
\end{itemize}

The script then:

\begin{itemize}
	\item Sleeps on a node for 30 seconds
	\item Uses the \tool{module} command to load the \texttt{gurobi/8.1.0} environment
	\item Prints the list of loaded modules into a file
\end{itemize}

The scheduler command, \tool{qsub}, is used to submit (non-interactive) jobs. 
From an ssh session on speed-submit, submit this job with \texttt{qsub ./tcsh.sh}. You will see,
\texttt{"Your job X ("qsub-test") has been submitted"}. The command, \tool{qstat}, can be used 
to look at the status of the cluster: \texttt{qstat -f -u "*"}. You will see 
something like this: 

\small
\begin{verbatim}
queuename                      qtype resv/used/tot. load_avg arch          states
---------------------------------------------------------------------------------
a.q@speed-01.encs.concordia.ca BIP   0/0/32         0.01     lx-amd64
---------------------------------------------------------------------------------
a.q@speed-03.encs.concordia.ca BIP   0/0/32         0.01     lx-amd64
---------------------------------------------------------------------------------
a.q@speed-25.encs.concordia.ca BIP   0/0/32         0.01     lx-amd64
---------------------------------------------------------------------------------
a.q@speed-27.encs.concordia.ca BIP   0/0/32         0.01     lx-amd64
---------------------------------------------------------------------------------
g.q@speed-05.encs.concordia.ca BIP   0/0/32         0.02     lx-amd64
     144   100.00000 qsub-test nul-uge     r     12/03/2018 16:39:30    1 
     62624 0.09843 case_talle x_yzabc      r     11/09/2021 16:50:09    32
---------------------------------------------------------------------------------
g.q@speed-17.encs.concordia.ca BIP   0/0/32         0.01     lx-amd64
---------------------------------------------------------------------------------
s.q@speed-07.encs.concordia.ca BIP   0/0/32         0.04     lx-amd64
---------------------------------------------------------------------------------
s.q@speed-08.encs.concordia.ca BIP   0/0/32         0.01     lx-amd64
---------------------------------------------------------------------------------
s.q@speed-09.encs.concordia.ca BIP   0/0/32         0.01     lx-amd64
---------------------------------------------------------------------------------
s.q@speed-10.encs.concordia.ca BIP   0/32/32        32.72    lx-amd64
     62624 0.09843 case_talle x_yzabc      r     11/09/2021 16:50:09    32
---------------------------------------------------------------------------------
s.q@speed-11.encs.concordia.ca BIP   0/32/32        32.08    lx-amd64
     62679 0.14212 CWLR_DF    a_bcdef      r     11/10/2021 17:25:19    32
---------------------------------------------------------------------------------
s.q@speed-12.encs.concordia.ca BIP   0/32/32        32.10    lx-amd64
     62749 0.09000 CLOUDY     z_abc        r     11/11/2021 21:58:12    32
---------------------------------------------------------------------------------
s.q@speed-15.encs.concordia.ca BIP   0/4/32         0.03     lx-amd64
     62753 82.47478 matlabLDPa b_bpxez      r     11/12/2021 08:49:52     4
---------------------------------------------------------------------------------
s.q@speed-16.encs.concordia.ca BIP   0/32/32        32.31    lx-amd64
     62751 0.09000 CLOUDY     z_abc        r     11/12/2021 06:03:54    32
---------------------------------------------------------------------------------
s.q@speed-19.encs.concordia.ca BIP   0/32/32        32.22    lx-amd64
---------------------------------------------------------------------------------
...
---------------------------------------------------------------------------------
s.q@speed-35.encs.concordia.ca BIP   0/32/32        2.78     lx-amd64
     62754 7.22952 qlogin-tes a_tiyuu      r     11/12/2021 10:31:06    32
---------------------------------------------------------------------------------
s.q@speed-36.encs.concordia.ca BIP   0/0/32         0.03     lx-amd64
etc.
\end{verbatim}
\normalsize

Remember that you only have 30 seconds before the job is essentially over, so 
if you do not see a similar output, either adjust the sleep time in the 
script, or execute the \tool{qstat} statement more quickly. The \tool{qstat} 
output listed above shows you that your job is 
running on node \texttt{speed-05}, that it has a job number of 144, that it 
was started at 16:39:30 on 12/03/2018, and that it is a single-core job (the 
default). 

Once the job finishes, there will be a new file in the directory that the job 
was started from, with the syntax of, \texttt{"job name".o"job number"}, so 
in this example the file is, qsub \file{test.o144}. This file represents the 
standard output (and error, if there is any) of the job in question. If you 
look at the contents of your newly created file, you will see that it 
contains the output of the, \texttt{module list} command. 
Important information is often written to this file.

Congratulations on your first job! 

% ------------------------------------------------------------------------------
\subsection{Common Job Management Commands Summary}
\label{sect:job-management-commands}

Here are useful job-management commands: 

\begin{itemize}
\item
\texttt{qsub ./<myscript>.sh}: once that your job script is ready,
on \texttt{speed-submit} you can submit it using this

\item
\texttt{qstat -f -u <ENCSusername>}: you can check the status of your job(s)

\item
\texttt{qstat -f -u "*"}: display cluster status for all users. 

\item
\texttt{qstat -j [job-ID]}: display job information for [job-ID] (said job may be actually running, or waiting in the queue). 

\item
\texttt{qdel [job-ID]}: delete job [job-ID]. 

\item
\texttt{qhold [job-ID]}: hold queued job, [job-ID], from running. 

\item
\texttt{qrls [job-ID]}: release held job [job-ID]. 

\item
\texttt{qacct -j [job-ID]}: get job stats. for completed job [job-ID]. \api{maxvmem} is one of the more useful stats. 
\end{itemize}


% ------------------------------------------------------------------------------
\subsection{Advanced \tool{qsub} Options}
\label{sect:qsub-options}

In addition to the basic \tool{qsub} options presented earlier, there are a 
few additional options that are generally useful:

\begin{itemize}
\item
\texttt{-m bea}: requests that the scheduler e-mail you when a job (b)egins;
(e)nds; (a)borts. Mail is sent to the default address of,
\texttt{"username@encs.concordia.ca"}, unless a different address is supplied (see, 
\texttt{-M}). The report sent when a job ends includes job 
runtime, as well as the maximum memory value hit (\api{maxvmem}). 

\item
\texttt{-M email@domain.com}: requests that the scheduler use this e-mail 
notification address, rather than the default (see, \texttt{-m}). 

\item
\texttt{-v variable[=value]}: exports an environment variable that can be used by the script.

\item
\texttt{-l h\_rt=[hour]:[min]:[sec]}: sets a job runtime of HH:MM:SS. Note 
that if you give a single number, that represents \emph{seconds}, not hours. 

\item
\texttt{-hold\_jid [job-ID]}: run this job only when job [job-ID] finishes. Held jobs appear in the queue. 
The many \tool{qsub} options available are read with, \texttt{man qsub}. Also 
note that \tool{qsub} options can be specified during the job-submission 
command, and these \emph{override} existing script options (if present). The 
syntax is, \texttt{qsub [options] PATHTOSCRIPT}, but unlike in the script, 
the options are specified without the leading \verb+#$+
(e.g., \texttt{qsub -N qsub-test -cwd -l h\_vmem=1G ./tcsh.sh}). 

\end{itemize}

% ------------------------------------------------------------------------------
\subsection{Array Jobs}

Array jobs are those that start a batch job or a parallel job multiple times. 
Each iteration of the job array is called a task and receives a unique job ID.

To submit an array job, use the \texttt{\-t} option of the \texttt{qsub} 
command as follows:

\begin{verbatim}
qsub -t n[-m[:s]] <batch_script>
\end{verbatim}

\textbf{-t Option Syntax:}
\begin{itemize}
\item
\texttt{n}: indicates the start-id.
\item
\texttt{m}: indicates the max-id.
\item
\texttt{s}: indicates the step size.
\end{itemize}

\textbf{Examples:}
\begin{itemize}
\item
\texttt{qsub -t 10 array.sh}: submits a job with 1 task where the task-id is 10. 
\item
\texttt{qsub -t 1-10 array.sh}: submits a job with 10 tasks numbered consecutively from 1 to 10.
\item
\texttt{qsub -t 3-15:3 array.sh}: submits a jobs with 5 tasks numbered consecutively with step size 3
(task-ids 3,6,9,12,15).
\end{itemize}

\textbf{Output files for Array Jobs:}

The default and output and error-files are \option{job\_name.[o|e]job\_id} and\\
\option{job\_name.[o|e]job\_id.task\_id}.
%
This means that Speed creates an output and an error-file for each task 
generated by the array-job as well as one for the super-ordinate array-job. 
To alter this behavior use the \option{-o} and \option{-e} option of
\tool{qsub}. 

For more details about Array Job options, please review the manual pages for 
\option{qsub} by executing the following at the command line on speed-submit 
\tool{man qsub}.
 
% ------------------------------------------------------------------------------
\subsection{Requesting Multiple Cores (i.e., Multithreading Jobs)}

For jobs that can take advantage of multiple machine cores, up to 32 cores
(per job) can be requested in your script with: 

\begin{verbatim}
#$ -pe smp [#cores] 
\end{verbatim}

\textbf{Do not request more cores than you think will be useful}, as larger-core
jobs are more difficult to schedule. On the flip side, though, if you 
are going to be running a program that scales out to the maximum single-machine
core count available, please (please) request 32 cores, to avoid node 
oversubscription (i.e., to avoid overloading the CPUs).

Core count associated with a job appears under, ``states'', in the,
\texttt{qstat -f -u "*"}, output.

% ------------------------------------------------------------------------------
\subsection{Interactive Jobs}

Job sessions can be interactive, instead of batch (script) based. Such 
sessions can be useful for testing and optimising code and resource 
requirements prior to batch submission. To request an interactive job 
session, use, \texttt{qlogin [options]}, similarly to a 
\tool{qsub} command-line job (e.g., \texttt{qlogin -N qlogin-test -l h\_vmem=1G}).
Note that the options that are available for \tool{qsub} are not necessarily
available for \tool{qlogin}, notably, \texttt{-cwd}, and, \texttt{-v}. 

% ------------------------------------------------------------------------------
\subsection{Scheduler Environment Variables}

The scheduler presents a number of environment variables that can be used in 
your jobs. Three of the more useful are \api{TMPDIR}, \api{SGE\_O\_WORKDIR}, 
and \api{NSLOTS}:

\begin{itemize}
\item
\api{\$TMPDIR}=the path to the job's temporary space on the node. It
\emph{only} exists for the duration of the job, so if data in the temporary space 
are important, they absolutely need to be accessed before the job terminates.

\item
\api{\$SGE\_O\_WORKDIR}=the path to the job's working directory (likely an
NFS-mounted path). If, \texttt{-cwd}, was stipulated, that path is taken; if not, 
the path defaults to your home directory.

\item
\api{\$NSLOTS}=the number of cores requested for the job. This variable can 
be used in place of hardcoded thread-request declarations. 

\end{itemize}

\noindent
In \xf{fig:tmpdir.sh} is a sample script, using all three.

\begin{figure}[htpb]
    \lstinputlisting[language=csh,frame=single,basicstyle=\footnotesize\ttfamily]{tmpdir.sh}
    \caption{Source code for \file{tmpdir.sh}}
	\label{fig:tmpdir.sh}
\end{figure}

% ------------------------------------------------------------------------------
\subsection{SSH Keys For MPI}

Some programs effect their parallel processing via MPI (which is a 
communication protocol). An example of such software is Fluent. MPI needs to 
have `passwordless login' set up, which means SSH keys. In your NFS-mounted 
home directory:

\begin{itemize}
\item
\texttt{cd .ssh}
\item
\texttt{ssh-keygen -t ed25519} (default location; blank passphrase) 
\item
\texttt{cat id\_ed25519.pub >> authorized\_keys} (if the \texttt{\href{https://www.ssh.com/academy/ssh/authorized-keys-file}{authorized\_keys}}
file already exists) \emph{OR} \texttt{cat id\_ed25519.pub > authorized\_keys} (if does not) 
\item
Set file permissions of \texttt{authorized\_keys} to 600; of your NFS-mounted home
to 700 (note that you likely will not have to do anything here, as most people
will have those permissions by default). 
\end{itemize}

% ------------------------------------------------------------------------------
\subsection{Creating Virtual Environments}
\label{sect:environments}

The following documentation is specific to the \textbf{Speed} HPC Facility at the
Gina Cody School of Engineering and Computer Science.

% ------------------------------------------------------------------------------
\subsubsection{Anaconda}

To create an anaconda environment in your speed-scratch directory, use the \texttt{\-\-prefix} 
option when executing \texttt{conda create}. For example, to create an anaconda environment for 
\texttt{ai\_user}, execute the following at the command line:

\begin{verbatim}
conda create --prefix /speed-scratch/a_user/myconda
\end{verbatim}

\vspace{10pt}
\noindent
\textbf{Note:} Without the \texttt{\-\-prefix} option, the \texttt{conda create} command creates the 
environment in texttt{a\_user}'s home directory by default.
\vspace{10pt}

% ------------------------------------------------------------------------------
\paragraph{List Environments.}

To view your conda environments, type: \texttt{conda info --envs}

\begin{verbatim}
# conda environments:
#
base                  *  /encs/pkg/anaconda3-2019.07/root
                         /speed-scratch/a_user/myconda
\end{verbatim}      

% ------------------------------------------------------------------------------
\paragraph{Activate an Environment.}

Activate the environment \texttt{\/speed\-scratch\/a\_user\/myconda} as follows
\begin{verbatim}
conda activate /speed-scratch/a_user/myconda
\end{verbatim}
After activating your environment, add \tool{pip} to your environment by using 
\begin{verbatim}
conda install pip
\end{verbatim}
This will install \tool{pip} and \tool{pip}'s dependencies, including python, 
into the environment.

\vspace{10pt}
\noindent
\textbf{Important Note:} \tool{pip} (and \tool{pip3}) are used to install modules
 from the python distribution while \texttt{conda install} installs modules from 
 anaconda's repository.
\vspace{10pt}


% ------------------------------------------------------------------------------
\subsection{Example Job Script: Fluent}

\begin{figure}[htpb]
    \lstinputlisting[language=csh,frame=single,basicstyle=\footnotesize\ttfamily]{fluent.sh}
    \caption{Source code for \file{fluent.sh}}
	\label{fig:fluent.sh}
\end{figure}

The job script in \xf{fig:fluent.sh} runs Fluent in parallel over 32 cores. 
Of note, we have requested e-mail notifications (\texttt{-m}), are defining the 
parallel environment for, \tool{fluent}, with, \texttt{-sgepe smp} (\textbf{very 
important}), and are setting \api{\$TMPDIR} as the in-job location for the
``moment'' \file{rfile.out} file (in-job, because the last line of the script 
copies everything from \api{\$TMPDIR} to a directory in the user's NFS-mounted home). 
Job progress can be monitored by examining the standard-out file (e.g.,
\file{flu10000.o249}), and/or by examining the ``moment'' file in 
\texttt{/disk/nobackup/<yourjob>} (hint: it starts with your job-ID) on the node running
the job. \textbf{Caveat:} take care with journal-file file paths.

% ------------------------------------------------------------------------------
\subsection{Example Job: efficientdet}

The following steps describing how to create an efficientdet environment on
\emph{Speed}, were submitted by a member of Dr. Amer's research group.

\begin{itemize}
    \item 
    Enter your ENCS user account's speed-scratch directory 
    \verb!cd /speed-scratch/<encs_username>!
    \item
    load python \verb!module load python/3.8.3!
    create virtual environment \verb!python3 -m venv <env_name>!
    activate virtual environment \verb!source <env_name>/bin/activate.csh!
    install DL packages for Efficientdet
\end{itemize}
\begin{verbatim}
pip install tensorflow==2.7.0
pip install lxml>=4.6.1
pip install absl-py>=0.10.0
pip install matplotlib>=3.0.3
pip install numpy>=1.19.4
pip install Pillow>=6.0.0
pip install PyYAML>=5.1
pip install six>=1.15.0
pip install tensorflow-addons>=0.12
pip install tensorflow-hub>=0.11
pip install neural-structured-learning>=1.3.1
pip install tensorflow-model-optimization>=0.5
pip install Cython>=0.29.13
pip install git+https://github.com/cocodataset/cocoapi.git#subdirectory=PythonAPI
\end{verbatim}

% ------------------------------------------------------------------------------
\subsection{Java Jobs}

Jobs that call \tool{java} have a memory overhead, which needs to be taken 
into account when assigning a value to \api{h\_vmem}. Even the most basic 
\tool{java} call, \texttt{java -Xmx1G -version}, will need to have,
\texttt{-l h\_vmem=5G}, with the 4-GB difference representing the memory overhead. 
Note that this memory overhead grows proportionally with the value of
\texttt{-Xmx}. To give you an idea, when \texttt{-Xmx} has a value of 100G,
\api{h\_vmem} has to be at least 106G; for 200G, at least 211G; for 300G, at least 314G.

% TODO: add a MARF Java job

% ------------------------------------------------------------------------------
\subsection{Scheduling On The GPU Nodes}

The primary cluster has two GPU nodes, each with six Tesla (CUDA-compatible) P6
cards: each card has 2048 cores and 16GB of RAM. Though note that the P6
is mainly a single-precision card, so unless you need the GPU double
precision, double-precision calculations will be faster on a CPU node.

Job scripts for the GPU queue differ in that they do not need these
statements:

\begin{verbatim}
#$ -pe smp <threadcount>
#$ -l h_vmem=<memory>G
\end{verbatim}

But do need this statement, which attaches either a single GPU, or, two
GPUs, to the job:

\begin{verbatim}
#$ -l gpu=[1|2]
\end{verbatim}

Single-GPU jobs are granted 5~CPU cores and 80GB of system memory, and
dual-GPU jobs are granted 10~CPU cores and 160GB of system memory. A
total of \emph{four} GPUs can be actively attached to any one user at any given
time.

Once that your job script is ready, you can submit it to the GPU queue
with:

\begin{verbatim}
qsub -q g.q ./<myscript>.sh
\end{verbatim}

And you can query \tool{nvidia-smi} on the node that is running your job with:

\begin{verbatim}
ssh <username>@speed[-05|-17] nvidia-smi
\end{verbatim}

Status of the GPU queue can be queried with:

\begin{verbatim}
qstat -f -u "*" -q g.q
\end{verbatim}

\textbf{Very important note} regarding TensorFlow and PyTorch: 
if you are planning to run TensorFlow and/or PyTorch multi-GPU jobs, 
do not use the \api{tf.distribute} and/or\\
\api{torch.nn.DataParallel} 
functions, as they will crash the compute node (100\% certainty). 
This appears to be the current hardware's architecture's defect.
%
The workaround is to either
% TODO: Need to link to that example
manually effect GPU parallelisation (TensorFlow has an example on how to
do this), or to run on a single GPU.

\vspace{10pt}
\noindent
\textbf{Important}
\vspace{10pt}

Users without permission to use the GPU nodes can submit jobs to the \texttt{g.q}
queue but those jobs will hang and never run.

There are two GPUs in both \texttt{speed-05} and \texttt{speed-17}, and one 
in \texttt{speed-19}. Their availability is seen with, \texttt{qstat -F g}
(note the capital): 

\small
\begin{verbatim}
queuename                      qtype resv/used/tot. load_avg arch          states
---------------------------------------------------------------------------------
...
---------------------------------------------------------------------------------
g.q@speed-05.encs.concordia.ca BIP   0/0/32         0.04     lx-amd64
        hc:gpu=6
---------------------------------------------------------------------------------
g.q@speed-17.encs.concordia.ca BIP   0/0/32         0.01     lx-amd64
        hc:gpu=6
---------------------------------------------------------------------------------
...
---------------------------------------------------------------------------------
s.q@speed-19.encs.concordia.ca BIP   0/32/32        32.37    lx-amd64
        hc:gpu=1
---------------------------------------------------------------------------------
etc. 
\end{verbatim}
\normalsize

This status demonstrates that all five are available (i.e., have not been 
requested as resources). To specifically request a GPU node, add,
\texttt{-l g=[\#GPUs]}, to your \tool{qsub} (statement/script) or
\tool{qlogin} (statement) request. For example,
\texttt{qsub -l h\_vmem=1G -l g=1 ./count.sh}. You 
will see that this job has been assigned to one of the GPU nodes:

\small
\begin{verbatim}
queuename                      qtype resv/used/tot. load_avg arch          states
--------------------------------------------------------------------------------- 
g.q@speed-05.encs.concordia.ca BIP 0/0/32 0.01 lx-amd64  hc:gpu=6 
--------------------------------------------------------------------------------- 
g.q@speed-17.encs.concordia.ca BIP 0/0/32 0.01 lx-amd64  hc:gpu=6 
--------------------------------------------------------------------------------- 
s.q@speed-19.encs.concordia.ca BIP 0/1/32 0.04 lx-amd64  hc:gpu=0 (haff=1.000000) 
       538 100.00000 count.sh   sbunnell     r     03/07/2019 02:39:39     1
---------------------------------------------------------------------------------
etc. 
\end{verbatim}
\normalsize

And that there are no more GPUs available on that node (\texttt{hc:gpu=0}). Note
that no more than two GPUs can be requested for any one job. 

% ------------------------------------------------------------------------------
\subsubsection{CUDA}

When calling \tool{CUDA} within job scripts, it is important to create a link to
the desired \tool{CUDA} libraries and set the runtime link path to the same libraries. 
For example, to use the \texttt{cuda-11.5} libraries, specify the following in 
your Makefile.

\begin{verbatim}
-L/encs/pkg/cuda-11.5/root/lib64 -Wl,-rpath,/encs/pkg/cuda-11.5/root/lib64
\end{verbatim}

In your job script, specify the version of \texttt{gcc} to use prior to calling 
cuda. For example: 
   \texttt{module load gcc/8.4}
or
   \texttt{module load gcc/9.3}

% ------------------------------------------------------------------------------
\subsubsection{Special Notes for sending CUDA jobs to the GPU Queue}

It is not possible to create a \texttt{qlogin} session on to a node in the 
\textbf{GPU Queue} (\texttt{g.q}). As direct logins to these nodes is not 
available, jobs must be submitted to the \textbf{GPU Queue} in order to compile 
and link.

We have several versions of CUDA installed in:
\begin{verbatim}
/encs/pkg/cuda-11.5/root/
/encs/pkg/cuda-10.2/root/
/encs/pkg/cuda-9.2/root
\end{verbatim}

For CUDA to compile properly for the GPU queue, edit your Makefile 
replacing \option{\/usr\/local\/cuda} with one of the above.


% ------------------------------------------------------------------------------
\section{Conclusion}
\label{sect:conclusion}

The cluster is, ``first come, first served'', until it fills, and then job
position in the queue is based upon past usage. The scheduler does attempt
to fill gaps, though, so sometimes a single-core job of lower priority
will schedule before a multi-core job of higher priority, for example.

% ------------------------------------------------------------------------------
\subsection{Important Limitations}
\label{sect:limitations}

\begin{itemize}
\item
New users are restricted to a total of 32 cores: write to \url{rt-ex-hpc@encs.concordia.ca}
if you need more temporarily (256 is the maximum possible, or, 8 jobs of 32 cores each).

\item
Job sessions are a maximum of one week in length (only 24 hours, though,
for interactive jobs).

\item
Scripts can live in your NFS-provided home, but any substantial data need
to be in your cluster-specific directory
(located at \verb+/speed-scratch/<ENCSusername>/+).

NFS is great for acute activity, but is not ideal for chronic activity.
Any data that a job will 
read more than once should be copied at the start to the scratch disk of a 
compute node using \api{\$TMPDIR} (and, perhaps, \api{\$SGE\_O\_WORKDIR}), 
any intermediary job data should be produced in \api{\$TMPDIR}, and once a 
job is near to finishing, those data should be copied to your NFS-mounted 
home (or other NFS-mounted space) from \api{\$TMPDIR} (to, perhaps,
\api{\$SGE\_O\_WORKDIR}). In other words, IO-intensive operations should be effected 
locally whenever possible, saving network activity for the start and end of 
jobs. 

\item
Your current resource allocation is based upon past usage, which is an 
amalgamation of approximately one week's worth of past wallclock (i.e., time 
spent on the node(s)) and CPU activity (on the node(s)).

\item
Jobs should NEVER be run outside of the province of the scheduler. Repeat 
offenders risk loss of cluster access. 

\end{itemize}

% ------------------------------------------------------------------------------
\subsection{Tips/Tricks}
\label{sect:tips}

\begin{itemize}
\item
Files/scripts must have Linux line breaks in them (not Windows ones).
\item
Use \tool{rsync}, not \tool{scp}, when moving data around. 
\item
If you are going to move many many files between NFS-mounted storage and the 
cluster, \tool{tar} everything up first. 
\item
If you intend to use a different shell (e.g., \tool{bash}~\cite{aosa-book-vol1-bash}),
you will need to source a different scheduler file, and will need to 
change the shell declaration in your script(s).
\item
The load displayed in \tool{qstat} by default is \api{np\_load}, which is
load/\#cores. That means that a load of, ``1'', which represents a fully active 
core, is displayed as $0.03$ on the node in question, as there are 32 cores 
on a node. To display load ``as is'' (such that a node with a fully active 
core displays a load of approximately $1.00$), add the following to your
\file{.tcshrc} file: \texttt{setenv SGE\_LOAD\_AVG load\_avg}

\item
Try to request resources that closely match what your job will use: 
requesting many more cores or much more memory than will be needed makes a 
job more difficult to schedule when resources are scarce.

\item
E-mail, \texttt{rt-ex-hpc AT encs.concordia.ca}, with any concerns/questions.
\end{itemize}

% ------------------------------------------------------------------------------
\subsection{Use Cases}
\label{sect:cases}

\begin{itemize}
	\item 
HPC Committee's initial batch about 6 students (end of 2019):
\begin{itemize}
	\item 
10000 iterations job in Fluent finished in $<26$ hours vs. 46 hours in Calcul Quebec
\end{itemize}
	\item 
NAG's MAC spoofer analyzer~\cite{mac-spoofer-analyzer-intro-c3s2e2014,mac-spoofer-analyzer-detail-fps2014},
such as \url{https://github.com/smokhov/atsm/tree/master/examples/flucid}
\begin{itemize}
	\item 
compilation of forensic computing reasoning cases about false or true positives of hardware address spoofing in the labs
\end{itemize}
	\item 
S4 LAB/GIPSY R\&D Group's:
\begin{itemize}
	\item 
MARFCAT and MARFPCAT (OSS signal processing and machine learning tools for 
vulnerable and weak code analysis and network packet capture
analysis)~\cite{marfcat-nlp-ai2014,marfcat-sate2010-nist,fingerprinting-mal-traffic}
	\item 
Web service data conversion and analysis
	\item 
{\flucid} encoders (translation of large log data into {\flucid}~\cite{mokhov-phd-thesis-2013} for forensic analysis)
	\item 
Genomic alignment exercises
\end{itemize}
\item
\bibentry{niksirat2020}
\end{itemize}

% ------------------------------------------------------------------------------
\appendix

% ------------------------------------------------------------------------------
\section{History}

% ------------------------------------------------------------------------------
\subsection{Acknowledgments}
\label{sect:scott-acks}

\begin{itemize}
	\item 
The first 6 versions of this manual and early job script samples,
Singularity testing and user support were produced/done by Dr.~Scott Bunnell
during his time at Concordia as a part of the NAG/HPC group. We thank
him for his contributions.
	\item 
The HTML version with devcontainer support was contributed by Anh H Nguyen.
\end{itemize}

% ------------------------------------------------------------------------------
\subsection{Phase 3}

Phase 3 had 4 vidpro nodes added from Dr.~Amer totalling 6x P6 and 6x V100
GPUs added.

% ------------------------------------------------------------------------------
\subsection{Phase 2}

Phase 2 saw 6x NVIDIA Tesla P6 added and 8x more compute nodes.
The P6s replaced 4x of FirePro S7150.

% ------------------------------------------------------------------------------
\subsection{Phase 1}

Phase 1 of Speed was of the following configuration:

\begin{itemize}
\item
Sixteen, 32-core nodes, each with 512~GB of memory and approximately 1~TB of volatile-scratch disk space. 
\item
Five AMD FirePro S7150 GPUs, with 8~GB of memory (compatible with the Direct X, OpenGL, OpenCL, and Vulkan APIs). 
\end{itemize}

% ------------------------------------------------------------------------------
\section{Frequently Asked Questions}
\label{sect:faqs}

% ------------------------------------------------------------------------------
\subsection{Where do I learn about linux? }
% ------------------------------------------------------------------------------
All Speed users are expected to have a basic understanding of Linux and its commonly used commands.
\subsubsection*{Software Carpentry}
Software Carpentry provides free resources to learn software, including a workshop on the Unix shell.
\url{https://software-carpentry.org/lessons/} 

\subsubsection*{Udemy}
There are a number of Udemy courses, including free ones, that will assist you in learning linux. Active Concordia faculty, staff and students have access to Udemy courses such as \textbf{Linux Mastery: Master the Linux Command Line in 11.5 Hours} is a good starting point for beginners. Visit \url{https://www.concordia.ca/it/services/udemy.html} to learn how Concordians may access Udemy.

% ------------------------------------------------------------------------------
\subsection{How to use the ``bash shell'' on Speed?}

This section describes how to use the ``bash shell'' on Speed. Review \autoref{sect:envsetup} to ensure that your bash enviroment is set up.
% ------------------------------------------------------------------------------

\subsubsection{How do I set bash as my login shell?}
In order to set your login shell to bash on Speed, your login shell on all GCS servers must be changed to bash.
To make this change, create a ticket with the Service Desk (or email help at concordia.ca) to request that bash become your default login shell for your ENCS user account on all GCS servers.

\subsubsection{How do I move into a bash shell on Speed?}
To move to the bash shell, type \textbf{bash} at the command prompt.
For example:
\begin{verbatim}
	[speed-27] [/home/a/a_user] > bash
	bash-4.4$ echo $0
	bash
\end{verbatim}	

Note how the command prompt changed from \verb![speed-27] [/home/a/a_user] >! to \verb!bash-4.4$! after entering the bash shell.

\subsubsection{How do I run scripts written in bash on Speed?}
To execute bash scripts on Speed:
\begin{enumerate}
	\item 
Ensure that the shebang of your bash job script is \verb!#!/encs/bin/bash!
	\item 
Use the qsub command to submit your job script to the scheduler.
\end{enumerate}

The Speed GitHub contains a sample \href{https://github.com/NAG-DevOps/speed-hpc/blob/master/src/bash.sh}{bash job script}.  

% ------------------------------------------------------------------------------
\subsection{How to resolve``Disk quota exceeded'' errors?}

% ------------------------------------------------------------------------------
\subsubsection{Probable Cause}

The \texttt{``Disk quota exceeded''} Error occurs when your application has run out of disk space to write to. On Speed this error can be returned when:
\begin{enumerate}
	\item
The \texttt{/tmp} directory on the speed node your application is running on is full and cannot be written to.
	\item
Your NFS-provided home is full and cannot be written to.
\end{enumerate}

% ------------------------------------------------------------------------------
\subsubsection{Possible Solutions}

\begin{enumerate}
	\item
Use the \textbf{-cwd} job script option to set the directory that the job script
 is submitted from the \texttt{job working directory}. The \texttt{job working directory} is the directory that the job will write output files in.
	\item
The use local disk space is generally recommended for IO intensive operations. However, as the size of \texttt{/tmp} on speed nodes 
is \texttt{1GB} it can be necessary for scripts to store temporary data 
elsewhere. 
Review the documentation for each module called within your script to 
determine how to set working directories for that application. 
The basic steps for this solution are:
\begin{itemize}
	\item
	Review the documentation on how to set working directories for 
	each module called by the job script.
	\item
	Create a working directory in speed-scratch for output files. 
	For example, this command will create a subdirectory called \textbf{output}
	 in your \verb!speed-scratch! directory:
	 \begin{verbatim}
		mkdir -m 750 /speed-scratch/$USER/output
	 \end{verbatim}
	\item
	To create a subdirectory for recovery files:
	\begin{verbatim}
		mkdir -m 750 /speed-scratch/$USER/recovery
	\end{verbatim}
	\item
	Update the job script to write output to the subdirectories you created in your \verb!speed-scratch! directory, e.g., \verb!/speed-scratch/$USER/output!.
	\end{itemize}
\end{enumerate}
In the above example, \verb!$USER! is an environment variable containing your ENCS username.
% ------------------------------------------------------------------------------
\subsubsection{Example of setting working directories for \tool{COMSOL}}

\begin{itemize}
	\item 
	Create directories for recovery, temporary, and configuration files. 
	For example, to create these directories for your encs user account:
	\begin{verbatim}
	mkdir -m 750 -p /speed-scratch/$USER/comsol/{recovery,tmp,config}
	\end{verbatim}
	\item
	Add the following command switches to the COMSOL command to use the 
	directories created above:
	\begin{verbatim} 
	-recoverydir /speed-scratch/$USER/comsol/recovery 
	-tmpdir /speed-scratch/$USER/comsol/tmp
	-configuration/speed-scratch/$USER/comsol/config
	\end{verbatim}
\end{itemize} 
In the above example, \verb!$USER! is an environment variable containing your ENCS username.
% ------------------------------------------------------------------------------
\subsubsection{Example of setting working directories for \tool{Python Modules}}

By default when adding a python module the \texttt{/tmp} directory is set as the temporary repository for files downloads. 
The size of the \texttt{/tmp} directory on \verb!speed-submit! is too small for pytorch.
To add a python module
\begin{itemize}
    \item 	
	Create your own tmp directory in your \verb!speed-scratch! direcrtory
	\begin{verbatim} 
        mkdir /speed-scratch/$USER/tmp
	\end{verbatim}
	\item
    Use the tmp direcrtory you created
	\begin{verbatim} 
        setenv TMPDIR /speed-scratch/$USER/tmp
	\end{verbatim}
    \item
	Attempt the installation of pytorch
\end{itemize}

In the above example, \verb!$USER! is an environment variable containing your ENCS username.
% ------------------------------------------------------------------------------
\section{Sister Facilities}

Below is a list of resources and facilities similar to Speed at various capacities.
Depending on your research group and needs, they might be available to you. They
are not managed by HPC/NAG of AITS, so contact their respective representatives.

\begin{itemize}
\item
\texttt{computation.encs} CPU only 3-machine cluster running longer jobs
without a scheduler
\item
\texttt{apini.encs} cluster for teaching and MPI programming (see the corresponding
course)
\item
Computer Science and Software Engineering (CSSE) Virya GPU Cluster
(2 nodes totalling 16 V100 NVIDIA GPUs), contact Alexander Aric at \texttt{gpu-help AT encs}
to request access if you are a CSSE member
\item
Dr. Maria Amer's VidPro group's nodes in Speed with additional V100 and P6 GPUs
(use \texttt{a.q} for those nodes).
\item
Dr. Amin Hammad's \texttt{construction.encs} Lambda Labs station
\item
Dr. Hassan Rivaz's \texttt{impactlab.encs} Lambda Labs station
\item
Dr. Ivan Contreras' servers
\item
Compute Canada / Calcul Quebec
\end{itemize}


% ------------------------------------------------------------------------------
% Refs:
%
\nocite{aosa-book-vol1}
\label{sect:bib}
%\bibliographystyle{IEEEtran}
\bibliographystyle{plain}
%\bibliographystyle{alpha}
%\bibliographystyle{unsrt}
%\bibliographystyle{abbrv}
% Create a section for references otherwise it appears to be part of the "Sister Facilities" Appendix
\clearpage
\addcontentsline{toc}{section}{Annotated Bibliography} 
\bibliography{speed-manual}

%------------------------------------------------------------------------------
\end{document}
