% ------------------------------------------------------------------------------
\section{Frequently Asked Questions}
\label{sect:faqs}

% ------------------------------------------------------------------------------
\subsection{Where do I learn about Linux?}

All Speed users are expected to have a basic understanding of Linux and its commonly used commands.

% ------------------------------------------------------------------------------
\subsubsection*{Software Carpentry}

Software Carpentry provides free resources to learn software, including a workshop on the Unix shell.
\url{https://software-carpentry.org/lessons/} 

% ------------------------------------------------------------------------------
\subsubsection*{Udemy}

There are a number of Udemy courses, including free ones, that will assist 
you in learning Linux. Active Concordia faculty, staff and students have 
access to Udemy courses such as \textbf{Linux Mastery: Master the Linux 
Command Line in 11.5 Hours} is a good starting point for beginners. Visit
\url{https://www.concordia.ca/it/services/udemy.html} to learn how Concordians 
may access Udemy.

% ------------------------------------------------------------------------------
\subsection{How to use the ``bash shell'' on Speed?}

This section describes how to use the ``bash shell'' on Speed. Review
\xs{sect:envsetup} to ensure that your bash environment is set up.

% ------------------------------------------------------------------------------
\subsubsection{How do I set bash as my login shell?}

In order to set your login shell to bash on Speed, your login shell on all GCS servers must be changed to bash.
To make this change, create a ticket with the Service Desk (or email \texttt{help at concordia.ca}) to
request that bash become your default login shell for your ENCS user account on all GCS servers.

% ------------------------------------------------------------------------------
\subsubsection{How do I move into a bash shell on Speed?}

To move to the bash shell, type \textbf{bash} at the command prompt.
For example:
\begin{verbatim}
	[speed-submit] [/home/a/a_user] > bash
	bash-4.4$ echo $0
	bash
\end{verbatim}	

Note how the command prompt changed from \verb![speed-submit] [/home/a/a_user] >! to \verb!bash-4.4$! after entering the bash shell.

If you use one of the below commands (make sure job request settings such
as memory, cores, etc are set), they will allocate your interactive
job sessions with \tool{bash} as a shell on the compute nodes:

\begin{itemize}
	\item \texttt{salloc ... /encs/bin/bash}
	\item \texttt{srun ... --pty /encs/bin/bash}
\end{itemize}

% ------------------------------------------------------------------------------
\subsubsection{How do I run scripts written in bash on Speed?}

To execute bash scripts on Speed:
\begin{enumerate}
	\item 
Ensure that the shebang of your bash job script is \verb+#!/encs/bin/bash+
	\item 
Use the \tool{sbatch} command to submit your job script to the scheduler.
\end{enumerate}

The Speed GitHub contains a sample
\href
  {https://github.com/NAG-DevOps/speed-hpc/blob/master/src/bash.sh}
	{bash job script}.

% ------------------------------------------------------------------------------
\subsection{How to resolve ``Disk quota exceeded'' errors?}

% ------------------------------------------------------------------------------
\subsubsection{Probable Cause}

The ``\texttt{Disk quota exceeded}'' Error occurs when your application has 
run out of disk space to write to. On Speed this error can be returned when:
 
\begin{enumerate}
	\item
Your NFS-provided home is full and cannot be written to.
You can verify this using \tool{quota} and \tool{bigfiles} commands.
	\item
The \texttt{/tmp} directory on the speed node your application is running on is full and cannot be written to.
\end{enumerate}

% ------------------------------------------------------------------------------
\subsubsection{Possible Solutions}

\begin{enumerate}
	\item
Use the \option{--chdir} job script option to set the directory that the job 
script is submitted from the \texttt{job working directory}. The
\texttt{job working directory} is the directory that the job will write output files in.
 	\item
The use local disk space is generally recommended for IO intensive operations.
However, as the size of \texttt{/tmp} on speed nodes 
is \texttt{1TB} it can be necessary for scripts to store temporary data 
elsewhere.
Review the documentation for each module called within your script to 
determine how to set working directories for that application. 
The basic steps for this solution are:
\begin{itemize}
	\item
	Review the documentation on how to set working directories for 
	each module called by the job script.
	\item
	Create a working directory in speed-scratch for output files. 
	For example, this command will create a subdirectory called \textbf{output}
	in your \verb!speed-scratch! directory:
	 \begin{verbatim}
		mkdir -m 750 /speed-scratch/$USER/output
	 \end{verbatim}
	\item
	To create a subdirectory for recovery files:
	\begin{verbatim}
		mkdir -m 750 /speed-scratch/$USER/recovery
	\end{verbatim}
	\item
	Update the job script to write output to the subdirectories you created in
	your \verb!speed-scratch! directory, e.g., \verb!/speed-scratch/$USER/output!.
	\end{itemize}
\end{enumerate}
In the above example, \verb!$USER! is an environment variable containing your ENCS username.

% ------------------------------------------------------------------------------
\subsubsection{Example of setting working directories for \tool{COMSOL}}

\begin{itemize}
	\item 
	Create directories for recovery, temporary, and configuration files. 
	For example, to create these directories for your GCS ENCS user account:
	\begin{verbatim}
	mkdir -m 750 -p /speed-scratch/$USER/comsol/{recovery,tmp,config}
	\end{verbatim}
	\item
	Add the following command switches to the COMSOL command to use the 
	directories created above:
	\begin{verbatim} 
	-recoverydir /speed-scratch/$USER/comsol/recovery 
	-tmpdir /speed-scratch/$USER/comsol/tmp
	-configuration/speed-scratch/$USER/comsol/config
	\end{verbatim}
\end{itemize} 

In the above example, \verb!$USER! is an environment variable containing your ENCS username.

% ------------------------------------------------------------------------------
\subsubsection{Example of setting working directories for \tool{Python Modules}}

By default when adding a python module the \texttt{/tmp} directory is set as the temporary repository for files downloads. 
The size of the \texttt{/tmp} directory on \verb!speed-submit! is too small for pytorch.
To add a python module
\begin{itemize}
    \item 	
	Create your own tmp directory in your \verb!speed-scratch! directory
	\begin{verbatim} 
  mkdir /speed-scratch/$USER/tmp
	\end{verbatim}
	\item
  Use the tmp directory you created
	\begin{verbatim} 
  setenv TMPDIR /speed-scratch/$USER/tmp
	\end{verbatim}
    \item
	Attempt the installation of pytorch
\end{itemize}

In the above example, \verb!$USER! is an environment variable containing your ENCS username.

% ------------------------------------------------------------------------------
\subsection{How do I check my job's status?}

%When a job with a job id of 1234 is running, the status of that job can be tracked using \verb!`qstat -j 1234`!.
%Likewise, if the job is pending, the \verb!`qstat -j 1234`! command will report as to why the job is not scheduled or running.
%Once the job has finished, or has been killed, the \textbf{qacct} command must be used to query the job's status, e.g., \verb!`qaact -j [jobid]`!. 
When a job with a job id of 1234 is running or terminated, the status of that job can be tracked using `\verb!sacct -j 1234!'.
\texttt{squeue -j 1234} can show while the job is sitting in the queue as well.
Long term statistics on the job after its terminated can be found using
\texttt{sstat -j 1234} after \tool{slurmctld} purges it its tracking state
into the database.

% ------------------------------------------------------------------------------
\subsection{Why is my job pending when nodes are empty?}

% ------------------------------------------------------------------------------
\subsubsection{Disabled nodes}

It is possible that a (or a number of) the Speed nodes are disabled. Nodes are disabled if they require maintenance. 
To verify if Speed nodes are disabled, see if they are in a draining or drained state:

%\begin{verbatim}
%qstat -f -qs d
%queuename                      qtype resv/used/tot. load_avg arch          states
%---------------------------------------------------------------------------------
%g.q@speed-05.encs.concordia.ca BIP   0/0/32         0.27     lx-amd64      d
%---------------------------------------------------------------------------------
%s.q@speed-07.encs.concordia.ca BIP   0/0/32         0.01     lx-amd64      d
%---------------------------------------------------------------------------------
%s.q@speed-10.encs.concordia.ca BIP   0/0/32         0.01     lx-amd64      d
%---------------------------------------------------------------------------------
%s.q@speed-16.encs.concordia.ca BIP   0/0/32         0.02     lx-amd64      d
%---------------------------------------------------------------------------------
%s.q@speed-19.encs.concordia.ca BIP   0/0/32         0.03     lx-amd64      d
%---------------------------------------------------------------------------------
%s.q@speed-24.encs.concordia.ca BIP   0/0/32         0.01     lx-amd64      d
%---------------------------------------------------------------------------------
%s.q@speed-36.encs.concordia.ca BIP   0/0/32         0.03     lx-amd64      d
%\end{verbatim}

\scriptsize
\begin{verbatim}
[serguei@speed-submit src] % sinfo --long --Node
Thu Oct 19 21:25:12 2023
NODELIST   NODES PARTITION       STATE CPUS    S:C:T MEMORY TMP_DISK WEIGHT AVAIL_FE REASON
speed-01       1        pa        idle 32     2:16:1 257458        0      1    gpu16 none
speed-03       1        pa        idle 32     2:16:1 257458        0      1    gpu32 none
speed-05       1        pg        idle 32     2:16:1 515490        0      1    gpu16 none
speed-07       1       ps*       mixed 32     2:16:1 515490        0      1    cpu32 none
speed-08       1       ps*     drained 32     2:16:1 515490        0      1    cpu32 UGE
speed-09       1       ps*     drained 32     2:16:1 515490        0      1    cpu32 UGE
speed-10       1       ps*     drained 32     2:16:1 515490        0      1    cpu32 UGE
speed-11       1       ps*        idle 32     2:16:1 515490        0      1    cpu32 none
speed-12       1       ps*     drained 32     2:16:1 515490        0      1    cpu32 UGE
speed-15       1       ps*     drained 32     2:16:1 515490        0      1    cpu32 UGE
speed-16       1       ps*     drained 32     2:16:1 515490        0      1    cpu32 UGE
speed-17       1        pg     drained 32     2:16:1 515490        0      1    gpu16 UGE
speed-19       1       ps*        idle 32     2:16:1 515490        0      1    cpu32 none
speed-20       1       ps*     drained 32     2:16:1 515490        0      1    cpu32 UGE
speed-21       1       ps*     drained 32     2:16:1 515490        0      1    cpu32 UGE
speed-22       1       ps*     drained 32     2:16:1 515490        0      1    cpu32 UGE
speed-23       1       ps*        idle 32     2:16:1 515490        0      1    cpu32 none
speed-24       1       ps*        idle 32     2:16:1 515490        0      1    cpu32 none
speed-25       1        pg        idle 32     2:16:1 257458        0      1    gpu32 none
speed-25       1        pa        idle 32     2:16:1 257458        0      1    gpu32 none
speed-27       1        pg        idle 32     2:16:1 257458        0      1    gpu32 none
speed-27       1        pa        idle 32     2:16:1 257458        0      1    gpu32 none
speed-29       1       ps*        idle 32     2:16:1 515490        0      1    cpu32 none
speed-30       1       ps*     drained 32     2:16:1 515490        0      1    cpu32 UGE
speed-31       1       ps*     drained 32     2:16:1 515490        0      1    cpu32 UGE
speed-32       1       ps*     drained 32     2:16:1 515490        0      1    cpu32 UGE
speed-33       1       ps*        idle 32     2:16:1 515490        0      1    cpu32 none
speed-34       1       ps*        idle 32     2:16:1 515490        0      1    cpu32 none
speed-35       1       ps*     drained 32     2:16:1 515490        0      1    cpu32 UGE
speed-36       1       ps*     drained 32     2:16:1 515490        0      1    cpu32 UGE
speed-37       1        pt        idle 256    2:64:2 980275        0      1 gpu20,mi none
speed-38       1        pt        idle 256    2:64:2 980275        0      1 gpu20,mi none
speed-39       1        pt        idle 256    2:64:2 980275        0      1 gpu20,mi none
speed-40       1        pt        idle 256    2:64:2 980275        0      1 gpu20,mi none
speed-41       1        pt        idle 256    2:64:2 980275        0      1 gpu20,mi none
speed-42       1        pt        idle 256    2:64:2 980275        0      1 gpu20,mi none
speed-43       1        pt        idle 256    2:64:2 980275        0      1 gpu20,mi none
\end{verbatim}
\normalsize

Note how the some of the Speed nodes in the above list have a state of \textbf{drained}
and the reason.
Your job will run once the maintenance has been completed and the disabled nodes have been enabled.

% ------------------------------------------------------------------------------
\subsubsection{Error in job submit request.}

It is possible that your job is pending, because the job requested resources that are not available within Speed.
To verify why pending job with job id 1234 is not running, execute `\verb!sacct -j 1234!'.
Summary of the reasons would also show up in \tool{squeue}.
%and review the messages in the \textbf{scheduling info:} section.
